% Copyright (C) 1997-1999, Ericsson Telecom AB
\documentstyle[]{report}
\begin{document}

\begin{center}

\vspace{12.0cm}

{\Large Literate Erlang 1.1}

\vspace{1.5cm}

Joe Armstrong\\
Computer Science Laboratory\\
Ericsson Telecom AB\\
\verb+joe@cslab.ericsson.se+\\
\bigskip
Richard Carlsson\\
Computing Science Department\\
Uppsala University\\
\verb+richardc@csd.uu.se+

\vspace{1.5cm}
%% Literate Erlang Report CS-JA-97-001\\
Last Revised \today
\vspace{2.5cm}

{\bf Abstract}

\end{center}

  This report describes \verb+EWEB+: a system for literate programming
in Erlang.

  A literate Erlang program consists of explanatory text alternating
with fragments of Erlang code. The fragments of code are written in an
order which best aids understanding of the program and not in the order
which they have to be presented to the Erlang compiler.

  The program \verb+etangle+ takes the code fragments and rearranges
them in a form which is acceptable to the Erlang compiler. \verb+eweave+
takes the program and formats it as a \LaTeX{} file. Both programs are
packaged together into \verb+EWEB+.

  \verb+EWEB+ follows the philosophy of \verb+noweb+ which was developed
by Norman Ramsey; \verb+noweb+ is a very much simplified version of
Donald Knuth's \verb+WEB+ system.

  This report contains the \verb+EWEB+ manual, instructions on how to
build the system, and the source of the program \verb+eweb.erl+.

\vspace{\fill}

\pagebreak

\section*{Introduction}

  This  report   is a literate Erlang  program.    The  program can be
compiled and  run,  like any other  program, or  printed.  You are now
reading the  printed version of   the program.  When you program  with
literate   Erlang you don't write    the program and the documentation
separately, you do both at the same time.   This approach has a number
of benefits.

  A literate Erlang program consists  of explanatory text  alternating
with fragments  of Erlang code.  The  fragments of code are written in
an order which best aids understanding  of the program  and not in the
order which they have to be presented to the Erlang compiler.

  The idea  of literate programming comes  from  from Donald E.  Knuth
The current  implementation which is called \verb+eweb+  is based on a
on \verb+noweb+  \cite{noweb}.  \verb+noweb+ was based  on \verb+cweb+
\cite{knuth} which was based on \verb+WEB+.

  To illustrate this we start  by describing the \verb+eweb+  program.
This   makes use of   a number   of small  utility  routines, such  as
\verb+str2tex+ which is defined as follows:

\begin{flushleft}
\label{utilities_start}
\noindent {\small {\sl $<<$utilities (1/8)$>>$} := $\bigtriangledown$ \pageref{utilities_2_8}}\\*
\tt
\noindent\makebox[0pt][r]{\makebox[4.5em][l]{1:}}%
\verb&str2tex([$_|T]) -> [$\\,$_|str2tex(T)];&\\
\noindent\makebox[0pt][r]{\makebox[4.5em][l]{2:}}%
\verb&str2tex([H|T])  -> [H|str2tex(T)];&\\
\noindent\makebox[0pt][r]{\makebox[4.5em][l]{3:}}%
\verb&str2tex([])     -> [].&\\
\end{flushleft}

  The annotation {\sl $<<$utilities  1/9$>>$} means that this  is part
one of six of a fragment called {\sl utilities\/} (the other fragments
are numbered 2/6,  3/6 etc.). \verb+:=+ means that  this is the  first
block   in the fragment.  $\bigtriangledown$   \pageref{utilities_2_8}
means that the next block is on page \pageref{utilities_2_8}.

Block two of {\sl utilities\/} contains the routine \verb+file2strings+:

\begin{flushleft}
\label{utilities_2_8}
\noindent {\small {\sl $<<$utilities (2/8)$>>$} += $\bigtriangleup$ \pageref{utilities_start} \ $\bigtriangledown$ \pageref{utilities_3_8}}\\*
\tt
\noindent\makebox[0pt][r]{\makebox[4.5em][l]{4:}}%
\verb&file2strings(File) ->&\\
\noindent\makebox[0pt][r]{\makebox[4.5em][l]{5:}}%
\verb&    case file:read_file(File) of&\\
\noindent\makebox[0pt][r]{\makebox[4.5em][l]{6:}}%
\verb&        {ok, Bin} ->&\\
\noindent\makebox[0pt][r]{\makebox[4.5em][l]{7:}}%
\verb&            binary2strings(Bin);&\\
\noindent\makebox[0pt][r]{\makebox[4.5em][l]{8:}}%
\verb&        {error, What} ->&\\
\noindent\makebox[0pt][r]{\makebox[4.5em][l]{9:}}%
\verb&            exit({misc,file2strings,File})&\\
\noindent\makebox[0pt][r]{\makebox[4.5em][l]{10:}}%
\verb&    end.&\\
\end{flushleft}

  \verb|+=| means that  this   is  a continuation of   the  previously
defined      fragment.   $\bigtriangleup$    \pageref{utilities_start}
$\bigtriangledown$  \pageref{utilities_3_8} means   that the  previous
part   of   the   {\sl      utilities\/}   fragment   was   on    page
\pageref{utilities_start},  and  that  the next  fragment    is  on  page
\pageref{utilities_3_8}.

Note also that the all lines of code are {\sl numbered}. This
is so that we can refer to lines of code within a given block of text.
So, for example we can make witty comments like:

{\sl 6} -- {\sl 9}: Test the return value of \verb+read_file+.

Line number references are added to the file as follows:

\begin{verbatim}
  <<Fragment name>> :=
  ...
  %#Ref
  ...
  >>
\end{verbatim}

\verb+%#Ref+ is a symbolic reference to the {\sl next} line in the
fragment. To refer to this line in the text use \verb+\L+\verb+N{Ref}+.
To typeset the line number in {\sl italics\/} use
\verb+\S+\verb+L{Ref}+.

\section*{Format of a literate Erlang program}

Literate Erlang programs (as of version 1.1) must start:
\begin{verbatim}
 # Any number of initial comment lines
 # (The `#' must be in the first column.)
 eweb1.1
 outdir=Dir
 outfiles=File1,File2,...
\end{verbatim}

    Here   \verb+Dir+   is   directory where   all  output   will  go.
\verb+File1, File2, ...+  is a comma separated   list of output  files
which will be created by eweb. If any of the files names \verb+File1+,
\verb+File2+   etc. begins  with  a plus   symbol (\verb|+|) then  the
execute bit of the file will be set.

There may not be any empty lines in the header. Any number of comment
lines (see below) may occur before the fields of the header, but not
between them. The fields must be specified in the above order.

The \verb+eweb+ version number in the header may be less than 1.1
(i.\,e., 1.0), in order to also be readable by older versions of this
program.

After the mandatory lines above, the following optional line may follow,
but can be left out, in which case an empty string is used for its
value. If the line does not match the below, the header is assumed to
end on the line before.
\begin{verbatim}
 copyright=...
\end{verbatim}

After the header follows arbitrary \LaTeX{} text, also containing code
blocks and possibly more weave file comments.

  Comments in a weave file (outside of code blocks) start with a
``\verb+#+'' character in column 1, and continue up until the end of the
line. Such lines are not included in the \LaTeX{} output.

\begin{verbatim}
 # This is a comment
\end{verbatim}

\subsection*{Code blocks}

At least each file name listed in \verb+outfiles+ must have a
correspondingly named code fragment.

The first block in a new fragment must be written:

\begin{verbatim}
 << Tag >> :=
        ... 
        ...
 >>
\end{verbatim}

Continuation blocks are written:

\begin{verbatim}
 << Tag >> +=
        ...
        ...
 >>
\end{verbatim}

  \verb+Tag+  is a an  arbitrary string which identifies the fragment.
\verb+Tag+ is delimited by the \verb+<<+ and \verb+>>+ symbols and can
contain any characters  (except  \verb+>>+). Note  that  the start tag
\verb+<<...>>+  and end symbols \verb+>>+  must be written starting in
column one of the program file.

    The special keyword \verb+hide+ (written in brackets) means that the
contents of the block should {\sl not\/} be written to the \LaTeX{}
output stream. For example:

\begin{verbatim}
    <<(hide)Tag>> =
        ...
        ...
    >>
\end{verbatim}

Means write the given text to \verb+File+ but {\bf not} to the generated
\LaTeX{} file. Note there are no spaces between the "(hide)" annotation
and the start of the tag.

Fragments can include other fragments, thus if we write:

\begin{verbatim}
 <<abc>> :=
        line 1
        <<more stuff here ...>>
        line2
 >>

 <<more stuff here ...>> :=
        hello
 >>
\end{verbatim}

Then \verb+etangle+ will expand the fragment \verb+abc+ into:

\begin{verbatim}
        line 1
        hello
        line 2
\end{verbatim}


\subsection*{Running the program}

  To run the program we evaluate the command \verb+eweb+ in the crrent
directory.

\subsection*{Bootstrapping the program}

    This  system is written  in itself {\sl  How did I  do that?} The
distribution contains two files -- \verb+eweb.lit+ and \verb+eweb.erl+
(which is generated from \verb+eweb.lit+).

    To  bootstrap  the  system  compile   the distributed version   of
\verb+eweb.erl+ with your favorite  Erlang compiler. Then evaluate
the expression:

\begin{verbatim}
        > eweb:file("eweb.lit").
\end{verbatim}

  If everything works (which it should) a new version of the file
\verb+eweb.erl+ will have been created which is identical to the file in
the distrbution, the file \verb+eweb.tex+ will also be created.
\verb+eweb.tex+ can be run through \LaTeX{} (twice, to get the page
numbers right) to get a pretty version of the documentation.

    Note if you {\sl change\/} this program remember to keep a copy of
the origonal which you can go back to if things go wrong!

\section*{Etangle and Eweave}

The overall structure of the program is as follows:

\begin{flushleft}
\label{eweb.erl_start}
\noindent {\small {\sl $<<$eweb.erl (1/1)$>>$} := }\\*
\tt
\noindent\makebox[0pt][r]{\makebox[4.5em][l]{11:}}%
\verb&-module(eweb).&\\
\noindent\makebox[0pt][r]{\makebox[4.5em][l]{12:}}%
\verb&-export([file/1, dir/0]).&\\
\noindent\makebox[0pt][r]{\makebox[4.5em][l]{13:}}%
\verb&&\\
\noindent\makebox[0pt][r]{\makebox[4.5em][l]{14:}}%
\verb&&{\sl $<<$imports \pageref{imports_start} $>>$}\\
\noindent\makebox[0pt][r]{\makebox[4.5em][l]{15:}}%
\verb&&{\sl $<<$main \pageref{main_start} $>>$}\\
\noindent\makebox[0pt][r]{\makebox[4.5em][l]{16:}}%
\verb&&{\sl $<<$parsing \pageref{parsing_start} $>>$}\\
\noindent\makebox[0pt][r]{\makebox[4.5em][l]{17:}}%
\verb&&{\sl $<<$analysis \pageref{analysis_start} $>>$}\\
\noindent\makebox[0pt][r]{\makebox[4.5em][l]{18:}}%
\verb&&{\sl $<<$etangle \pageref{etangle_start} $>>$}\\
\noindent\makebox[0pt][r]{\makebox[4.5em][l]{19:}}%
\verb&&{\sl $<<$eweave \pageref{eweave_start} $>>$}\\
\noindent\makebox[0pt][r]{\makebox[4.5em][l]{20:}}%
\verb&&{\sl $<<$error handling \pageref{error handling_start} $>>$}\\
\noindent\makebox[0pt][r]{\makebox[4.5em][l]{21:}}%
\verb&&{\sl $<<$utilities \pageref{utilities_start} $>>$}\\
\end{flushleft}

{\sl 14} -- {\sl 21}: Include a number of fragments which make up the
program. The fragment {\sl imports}, for example, starts on pag
\pageref{imports_start}.

    We start with \verb+file/1+ which  is the main  entry point of the
module.  Most of file is enclosed in a big  \verb+catch begin ... end+
construct:

\begin{flushleft}
\label{main_start}
\noindent {\small {\sl $<<$main (1/1)$>>$} := }\\*
\tt
\noindent\makebox[0pt][r]{\makebox[4.5em][l]{22:}}%
\verb&file(File) ->&\\
\noindent\makebox[0pt][r]{\makebox[4.5em][l]{23:}}%
\verb&    LitFile = File ++ ".lit",&\\
\noindent\makebox[0pt][r]{\makebox[4.5em][l]{24:}}%
\verb&    io:fwrite("Processing:~p~n", [LitFile]),&\\
\noindent\makebox[0pt][r]{\makebox[4.5em][l]{25:}}%
\verb&    case (catch begin&\\
\noindent\makebox[0pt][r]{\makebox[4.5em][l]{26:}}%
\verb&          clear_errors(),&\\
\noindent\makebox[0pt][r]{\makebox[4.5em][l]{27:}}%
\verb&          Strings = read_file(LitFile),&\\
\noindent\makebox[0pt][r]{\makebox[4.5em][l]{28:}}%
\verb&          throw_if_errors(),&\\
\noindent\makebox[0pt][r]{\makebox[4.5em][l]{29:}}%
\verb&          {OutDir, OutFiles, Copy, Blocks0} = &\\
\noindent\makebox[0pt][r]{\makebox[4.5em][l]{30:}}%
\verb&              strings2blocks(Strings),&\\
\noindent\makebox[0pt][r]{\makebox[4.5em][l]{31:}}%
\verb&          throw_if_errors(),&\\
\noindent\makebox[0pt][r]{\makebox[4.5em][l]{32:}}%
\verb&          Blocks1 = number_blocks(Blocks0),&\\
\noindent\makebox[0pt][r]{\makebox[4.5em][l]{33:}}%
\verb&          throw_if_errors(),&\\
\noindent\makebox[0pt][r]{\makebox[4.5em][l]{34:}}%
\verb&          {Blocks, Dict} = number_lines(Blocks1),&\\
\noindent\makebox[0pt][r]{\makebox[4.5em][l]{35:}}%
\verb&          throw_if_errors(),&\\
\noindent\makebox[0pt][r]{\makebox[4.5em][l]{36:}}%
\verb&          eweave(OutDir, File, Blocks, Dict),&\\
\noindent\makebox[0pt][r]{\makebox[4.5em][l]{37:}}%
\verb&          etangle(File, OutDir, OutFiles, Copy, Blocks)&\\
\noindent\makebox[0pt][r]{\makebox[4.5em][l]{38:}}%
\verb&          end) of&\\
\noindent\makebox[0pt][r]{\makebox[4.5em][l]{39:}}%
\verb&        {errors, E} ->&\\
\noindent\makebox[0pt][r]{\makebox[4.5em][l]{40:}}%
\verb&            report_errors(E),&\\
\noindent\makebox[0pt][r]{\makebox[4.5em][l]{41:}}%
\verb&            error;&\\
\noindent\makebox[0pt][r]{\makebox[4.5em][l]{42:}}%
\verb&        {'EXIT', Why} ->&\\
\noindent\makebox[0pt][r]{\makebox[4.5em][l]{43:}}%
\verb&            io:fwrite("EXIT ~p~n", [Why]),&\\
\noindent\makebox[0pt][r]{\makebox[4.5em][l]{44:}}%
\verb&            error;&\\
\noindent\makebox[0pt][r]{\makebox[4.5em][l]{45:}}%
\verb&        Other ->&\\
\noindent\makebox[0pt][r]{\makebox[4.5em][l]{46:}}%
\verb&            Other&\\
\noindent\makebox[0pt][r]{\makebox[4.5em][l]{47:}}%
\verb&    end.&\\
\end{flushleft}

{\sl 37}: Note the {\sl return\/} value of \verb+file+ is whatever
\verb+etangle+ returns and this is a list of files which were created,
this can be useful for combining \verb+eweb+ with other programs.

Error handling in this program is slightly complicated since
we want to collect {\sl all\/} the errors in a given pass
and not just abort on the first error.
The error handling routines use the global variable \verb+error$list+.
which is used to stored a list of all errors which have occured.
This is at the start of the program by calling:

\begin{flushleft}
\label{error handling_start}
\noindent {\small {\sl $<<$error handling (1/5)$>>$} := $\bigtriangledown$ \pageref{error handling_2_5}}\\*
\tt
\noindent\makebox[0pt][r]{\makebox[4.5em][l]{48:}}%
\verb&clear_errors() -> put('error$list', []).&\\
\end{flushleft}

If an error occurs {\sl within} a pass we call \verb+error(X)+ 
which adds the error to the error list.

\begin{flushleft}
\label{error handling_2_5}
\noindent {\small {\sl $<<$error handling (2/5)$>>$} += $\bigtriangleup$ \pageref{error handling_start} \ $\bigtriangledown$ \pageref{error handling_3_5}}\\*
\tt
\noindent\makebox[0pt][r]{\makebox[4.5em][l]{49:}}%
\verb&error(E) -> &\\
\noindent\makebox[0pt][r]{\makebox[4.5em][l]{50:}}%
\verb&    put('error$list', [E|get('error$list')]), error.&\\
\end{flushleft}

  Between each pass  \verb+throw_if_errors+ is called.  This checks if
any errors have been reported.   If any errors   have occured then  an
exception is raised which will be caught in \verb+file/1+.

\begin{flushleft}
\label{error handling_3_5}
\noindent {\small {\sl $<<$error handling (3/5)$>>$} += $\bigtriangleup$ \pageref{error handling_2_5} \ $\bigtriangledown$ \pageref{error handling_4_5}}\\*
\tt
\noindent\makebox[0pt][r]{\makebox[4.5em][l]{51:}}%
\verb&throw_if_errors() ->&\\
\noindent\makebox[0pt][r]{\makebox[4.5em][l]{52:}}%
\verb&    case get('error$list') of&\\
\noindent\makebox[0pt][r]{\makebox[4.5em][l]{53:}}%
\verb&        [] ->&\\
\noindent\makebox[0pt][r]{\makebox[4.5em][l]{54:}}%
\verb&            true;&\\
\noindent\makebox[0pt][r]{\makebox[4.5em][l]{55:}}%
\verb&        Errors ->&\\
\noindent\makebox[0pt][r]{\makebox[4.5em][l]{56:}}%
\verb&            throw({errors, reverse(Errors)})&\\
\noindent\makebox[0pt][r]{\makebox[4.5em][l]{57:}}%
\verb&    end.&\\
\end{flushleft}

  Note  we call \verb+reverse(Errors)+  to reverse the order of errors
in the error list.  This is because errors are reported by adding them
to the head of the error list. When we report  errors we want to do so
in the order in which they occured. Reporting errors is done with:

\begin{flushleft}
\label{error handling_4_5}
\noindent {\small {\sl $<<$error handling (4/5)$>>$} += $\bigtriangleup$ \pageref{error handling_3_5} \ $\bigtriangledown$ \pageref{error handling_5_5}}\\*
\tt
\noindent\makebox[0pt][r]{\makebox[4.5em][l]{58:}}%
\verb&report_errors(Errors) ->&\\
\noindent\makebox[0pt][r]{\makebox[4.5em][l]{59:}}%
\verb&    map(fun display_error/1, Errors).&\\
\noindent\makebox[0pt][r]{\makebox[4.5em][l]{60:}}%
\verb&&\\
\noindent\makebox[0pt][r]{\makebox[4.5em][l]{61:}}%
\verb&display_error(X) ->&\\
\noindent\makebox[0pt][r]{\makebox[4.5em][l]{62:}}%
\verb&    io:fwrite("Error:~p~n", [X]).&\\
\end{flushleft}

Finally, we may want to abort {\sl immediately} if a serious error occurs:

\begin{flushleft}
\label{error handling_5_5}
\noindent {\small {\sl $<<$error handling (5/5)$>>$} += $\bigtriangleup$ \pageref{error handling_4_5}}\\*
\tt
\noindent\makebox[0pt][r]{\makebox[4.5em][l]{63:}}%
\verb&fatal(Error) ->&\\
\noindent\makebox[0pt][r]{\makebox[4.5em][l]{64:}}%
\verb&    throw({errors, [Error]}).&\\
\end{flushleft}

  \verb+read_file(File)+  reads a file  by calling \verb+file2strings+
which returns a list   of the form  \verb+[{LineNo, String}]+.   When
processing text files we often need line numbers for reporting errors,
it proves convenient to store the  lines number together with the data
on the line in a single data structure for ease of future reference.

\begin{flushleft}
\label{parsing_start}
\noindent {\small {\sl $<<$parsing (1/13)$>>$} := $\bigtriangledown$ \pageref{parsing_2_13}}\\*
\tt
\noindent\makebox[0pt][r]{\makebox[4.5em][l]{65:}}%
\verb&read_file(File) ->&\\
\noindent\makebox[0pt][r]{\makebox[4.5em][l]{66:}}%
\verb&    case catch file2strings(File) of&\\
\noindent\makebox[0pt][r]{\makebox[4.5em][l]{67:}}%
\verb&        {'EXIT', Why}->&\\
\noindent\makebox[0pt][r]{\makebox[4.5em][l]{68:}}%
\verb&            error({cannot_read,file,File,Why});&\\
\noindent\makebox[0pt][r]{\makebox[4.5em][l]{69:}}%
\verb&        Strings ->&\\
\noindent\makebox[0pt][r]{\makebox[4.5em][l]{70:}}%
\verb&            Strings&\\
\noindent\makebox[0pt][r]{\makebox[4.5em][l]{71:}}%
\verb&    end.&\\
\end{flushleft}

  Having read the file, we parse it. \verb+strings2blocks+ checks that
the file has a correct header. An incorrect header generates a fatal
error.

\begin{flushleft}
\label{parsing_2_13}
\noindent {\small {\sl $<<$parsing (2/13)$>>$} += $\bigtriangleup$ \pageref{parsing_start} \ $\bigtriangledown$ \pageref{parsing_3_13}}\\*
\tt
\noindent\makebox[0pt][r]{\makebox[4.5em][l]{72:}}%
\verb&strings2blocks([{_, [$# | _]} | T]) ->&\\
\noindent\makebox[0pt][r]{\makebox[4.5em][l]{73:}}%
\verb&    strings2blocks(T);&\\
\noindent\makebox[0pt][r]{\makebox[4.5em][l]{74:}}%
\verb&strings2blocks([{L, [$e, $w, $e, $b | S]} | T]) ->&\\
\noindent\makebox[0pt][r]{\makebox[4.5em][l]{75:}}%
\verb&    parse_header([{L, [$e, $w, $e, $b | S]} | T]);&\\
\noindent\makebox[0pt][r]{\makebox[4.5em][l]{76:}}%
\verb&strings2blocks(_) ->&\\
\noindent\makebox[0pt][r]{\makebox[4.5em][l]{77:}}%
\verb&    fatal({file,has,incorrect,header}).&\\
\noindent\makebox[0pt][r]{\makebox[4.5em][l]{78:}}%
\verb&&\\
\noindent\makebox[0pt][r]{\makebox[4.5em][l]{79:}}%
\verb&parse_header([{_,[$e,$w,$e,$b,$1,$.,$0|_]} | T]) ->&\\
\noindent\makebox[0pt][r]{\makebox[4.5em][l]{80:}}%
\verb&    parse_header_1(T);&\\
\noindent\makebox[0pt][r]{\makebox[4.5em][l]{81:}}%
\verb&parse_header([{_,[$e,$w,$e,$b,$1,$.,$1|_]} | T]) ->&\\
\noindent\makebox[0pt][r]{\makebox[4.5em][l]{82:}}%
\verb&    parse_header_1(T);&\\
\noindent\makebox[0pt][r]{\makebox[4.5em][l]{83:}}%
\verb&parse_header(_) ->&\\
\noindent\makebox[0pt][r]{\makebox[4.5em][l]{84:}}%
\verb&    fatal({file,has,incorrect,header}).&\\
\noindent\makebox[0pt][r]{\makebox[4.5em][l]{85:}}%
\verb&&\\
\noindent\makebox[0pt][r]{\makebox[4.5em][l]{86:}}%
\verb&parse_header_1([{_,[$o,$u,$t,$d,$i,$r,$=|OutDir]},&\\
\noindent\makebox[0pt][r]{\makebox[4.5em][l]{87:}}%
\verb&                {_,[$o,$u,$t,$f,$i,$l,$e,$s,$=|OutFiles]}&\\
\noindent\makebox[0pt][r]{\makebox[4.5em][l]{88:}}%
\verb&                |T]) ->&\\
\noindent\makebox[0pt][r]{\makebox[4.5em][l]{89:}}%
\verb&    {Copyright, Blocks} = parse_options(T),&\\
\noindent\makebox[0pt][r]{\makebox[4.5em][l]{90:}}%
\verb&    {OutDir, OutFiles, Copyright, Blocks};&\\
\noindent\makebox[0pt][r]{\makebox[4.5em][l]{91:}}%
\verb&parse_header_1(_) ->&\\
\noindent\makebox[0pt][r]{\makebox[4.5em][l]{92:}}%
\verb&    fatal({file,has,incorrect,header}).&\\
\end{flushleft}

This handles the optional header fields (these are still required to be
specified in order).

\begin{flushleft}
\label{parsing_3_13}
\noindent {\small {\sl $<<$parsing (3/13)$>>$} += $\bigtriangleup$ \pageref{parsing_2_13} \ $\bigtriangledown$ \pageref{parsing_4_13}}\\*
\tt
\noindent\makebox[0pt][r]{\makebox[4.5em][l]{93:}}%
\verb&parse_options([{_,[$c,$o,$p,$y,$r,$i,$g,$h,$t,$=|C]}&\\
\noindent\makebox[0pt][r]{\makebox[4.5em][l]{94:}}%
\verb&               | T]) ->&\\
\noindent\makebox[0pt][r]{\makebox[4.5em][l]{95:}}%
\verb&        {C, parse1(T)};&\\
\noindent\makebox[0pt][r]{\makebox[4.5em][l]{96:}}%
\verb&parse_options(T) ->&\\
\noindent\makebox[0pt][r]{\makebox[4.5em][l]{97:}}%
\verb&        {"", parse1(T)}.&\\
\end{flushleft}

    \verb+parse1+ collects together \LaTeX{} blocks and the blocks which
belong to a fragment.

\begin{flushleft}
\label{parsing_4_13}
\noindent {\small {\sl $<<$parsing (4/13)$>>$} += $\bigtriangleup$ \pageref{parsing_3_13} \ $\bigtriangledown$ \pageref{parsing_5_13}}\\*
\tt
\noindent\makebox[0pt][r]{\makebox[4.5em][l]{98:}}%
\verb&parse1(T) -> parse1(T, []).&\\
\noindent\makebox[0pt][r]{\makebox[4.5em][l]{99:}}%
\verb&&\\
\noindent\makebox[0pt][r]{\makebox[4.5em][l]{100:}}%
\verb&parse1([], Blocks) ->&\\
\noindent\makebox[0pt][r]{\makebox[4.5em][l]{101:}}%
\verb&    reverse(Blocks);&\\
\noindent\makebox[0pt][r]{\makebox[4.5em][l]{102:}}%
\verb&parse1(T, Blocks) ->&\\
\noindent\makebox[0pt][r]{\makebox[4.5em][l]{103:}}%
\verb&    case get_block(T) of&\\
\noindent\makebox[0pt][r]{\makebox[4.5em][l]{104:}}%
\verb&        none ->&\\
\noindent\makebox[0pt][r]{\makebox[4.5em][l]{105:}}%
\verb&            reverse(Blocks);&\\
\noindent\makebox[0pt][r]{\makebox[4.5em][l]{106:}}%
\verb&        {B, T1} ->&\\
\noindent\makebox[0pt][r]{\makebox[4.5em][l]{107:}}%
\verb&            parse1(T1, [B|Blocks])&\\
\noindent\makebox[0pt][r]{\makebox[4.5em][l]{108:}}%
\verb&    end.&\\
\end{flushleft}

  \verb+get_block+ returns \verb+none+ if there are no more blocks, or
\verb+{Block, Rest}+ where \verb+Block+      is the next  block    and
\verb+Rest+ are the  remaining lines {\sl  after\/} the current block.
\verb+get_block+   calls  either   \verb+get_tagged_block_body+     or
\verb+get_untagged+  block to do  the work of collecting the different
block types.

\begin{flushleft}
\label{parsing_5_13}
\noindent {\small {\sl $<<$parsing (5/13)$>>$} += $\bigtriangleup$ \pageref{parsing_4_13} \ $\bigtriangledown$ \pageref{parsing_6_13}}\\*
\tt
\noindent\makebox[0pt][r]{\makebox[4.5em][l]{109:}}%
\verb&get_block([]) ->&\\
\noindent\makebox[0pt][r]{\makebox[4.5em][l]{110:}}%
\verb&    none;&\\
\noindent\makebox[0pt][r]{\makebox[4.5em][l]{111:}}%
\verb&get_block([{Line,[$<,$<|T1]}|T2]) ->&\\
\noindent\makebox[0pt][r]{\makebox[4.5em][l]{112:}}%
\verb&    Header = parse_block_header(Line, [$<,$<|T1]),&\\
\noindent\makebox[0pt][r]{\makebox[4.5em][l]{113:}}%
\verb&    {Lines, T3} = get_tagged_block_body(T2, []),&\\
\noindent\makebox[0pt][r]{\makebox[4.5em][l]{114:}}%
\verb&    {{block, Line, Header, Lines}, T3};&\\
\noindent\makebox[0pt][r]{\makebox[4.5em][l]{115:}}%
\verb&get_block([H|T]) ->&\\
\noindent\makebox[0pt][r]{\makebox[4.5em][l]{116:}}%
\verb&    {Lines, T1} = get_untagged_block(T, [H]),&\\
\noindent\makebox[0pt][r]{\makebox[4.5em][l]{117:}}%
\verb&    {{tex, Lines}, T1}.&\\
\end{flushleft}

  \verb+get_tagged_block_body+ collects  lines until  a line  starting
\verb+>>+ is encountered or until there are no more input lines.

\begin{flushleft}
\label{parsing_6_13}
\noindent {\small {\sl $<<$parsing (6/13)$>>$} += $\bigtriangleup$ \pageref{parsing_5_13} \ $\bigtriangledown$ \pageref{parsing_7_13}}\\*
\tt
\noindent\makebox[0pt][r]{\makebox[4.5em][l]{118:}}%
\verb&get_tagged_block_body([], Lines) ->&\\
\noindent\makebox[0pt][r]{\makebox[4.5em][l]{119:}}%
\verb&    {reverse(Lines), []};&\\
\noindent\makebox[0pt][r]{\makebox[4.5em][l]{120:}}%
\verb&get_tagged_block_body([{_,[$>,$>|_]}|T], Lines) ->&\\
\noindent\makebox[0pt][r]{\makebox[4.5em][l]{121:}}%
\verb&    {reverse(Lines), T};&\\
\noindent\makebox[0pt][r]{\makebox[4.5em][l]{122:}}%
\verb&get_tagged_block_body([H|T], Lines) ->&\\
\noindent\makebox[0pt][r]{\makebox[4.5em][l]{123:}}%
\verb&    get_tagged_block_body(T, [parse_data(H)|Lines]).&\\
\end{flushleft}

\verb+get_untagged_block+ collects lines until a line starting
\verb+<<+ is seen:
 
\begin{flushleft}
\label{parsing_7_13}
\noindent {\small {\sl $<<$parsing (7/13)$>>$} += $\bigtriangleup$ \pageref{parsing_6_13} \ $\bigtriangledown$ \pageref{parsing_8_13}}\\*
\tt
\noindent\makebox[0pt][r]{\makebox[4.5em][l]{124:}}%
\verb&get_untagged_block([H|T], Lines) ->&\\
\noindent\makebox[0pt][r]{\makebox[4.5em][l]{125:}}%
\verb&    case H of&\\
\noindent\makebox[0pt][r]{\makebox[4.5em][l]{126:}}%
\verb&        {_Line,[$<,$<|_]} ->&\\
\noindent\makebox[0pt][r]{\makebox[4.5em][l]{127:}}%
\verb&            {reverse(Lines), [H|T]};&\\
\noindent\makebox[0pt][r]{\makebox[4.5em][l]{128:}}%
\verb&        _ ->&\\
\noindent\makebox[0pt][r]{\makebox[4.5em][l]{129:}}%
\verb&            get_untagged_block(T, [H|Lines])&\\
\noindent\makebox[0pt][r]{\makebox[4.5em][l]{130:}}%
\verb&    end;&\\
\noindent\makebox[0pt][r]{\makebox[4.5em][l]{131:}}%
\verb&get_untagged_block([], Lines) ->&\\
\noindent\makebox[0pt][r]{\makebox[4.5em][l]{132:}}%
\verb&    {reverse(Lines), []}.&\\
\end{flushleft}

Event time a block starting \verb+<<+ is seen \verb+parse_block_header+ is
called to parse the contents of the header:

\begin{flushleft}
\label{parsing_8_13}
\noindent {\small {\sl $<<$parsing (8/13)$>>$} += $\bigtriangleup$ \pageref{parsing_7_13} \ $\bigtriangledown$ \pageref{parsing_9_13}}\\*
\tt
\noindent\makebox[0pt][r]{\makebox[4.5em][l]{133:}}%
\verb&parse_block_header(Line, Str) ->&\\
\noindent\makebox[0pt][r]{\makebox[4.5em][l]{134:}}%
\verb&    case parse_tag(Str) of&\\
\noindent\makebox[0pt][r]{\makebox[4.5em][l]{135:}}%
\verb&        {ok, Hide, Tag, Str1} ->&\\
\noindent\makebox[0pt][r]{\makebox[4.5em][l]{136:}}%
\verb&            case parse_after_tag(Str1) of&\\
\noindent\makebox[0pt][r]{\makebox[4.5em][l]{137:}}%
\verb&                {ok, Type} ->&\\
\noindent\makebox[0pt][r]{\makebox[4.5em][l]{138:}}%
\verb&                    {Hide, Tag, Type};&\\
\noindent\makebox[0pt][r]{\makebox[4.5em][l]{139:}}%
\verb&                error ->&\\
\noindent\makebox[0pt][r]{\makebox[4.5em][l]{140:}}%
\verb&                    error({badTagFollower,line,Line,&\\
\noindent\makebox[0pt][r]{\makebox[4.5em][l]{141:}}%
\verb&                           must,be,'+=','or',':='})&\\
\noindent\makebox[0pt][r]{\makebox[4.5em][l]{142:}}%
\verb&            end;&\\
\noindent\makebox[0pt][r]{\makebox[4.5em][l]{143:}}%
\verb&        error ->&\\
\noindent\makebox[0pt][r]{\makebox[4.5em][l]{144:}}%
\verb&            error({badTag,line,Line,is,Str})&\\
\noindent\makebox[0pt][r]{\makebox[4.5em][l]{145:}}%
\verb&    end.&\\
\end{flushleft}

Tags are assumed to look like \verb+<<(hide)Tag>>+

    \verb+parse_tag(Str)+ is used for tag parsing.  It returns
\verb+{ok,Hide,Tag,Rest}+ or \verb+error+.  \verb+Hide+ is \verb+hide+
if the block should be hidden or \verb+show+ if the block should be
shown.

\begin{flushleft}
\label{parsing_9_13}
\noindent {\small {\sl $<<$parsing (9/13)$>>$} += $\bigtriangleup$ \pageref{parsing_8_13} \ $\bigtriangledown$ \pageref{parsing_10_13}}\\*
\tt
\noindent\makebox[0pt][r]{\makebox[4.5em][l]{146:}}%
\verb&parse_tag(Str1) ->&\\
\noindent\makebox[0pt][r]{\makebox[4.5em][l]{147:}}%
\verb&    case skip_white_space(Str1) of&\\
\noindent\makebox[0pt][r]{\makebox[4.5em][l]{148:}}%
\verb&        [$<,$<,$(,$h,$i,$d,$e,$)|Str2] ->&\\
\noindent\makebox[0pt][r]{\makebox[4.5em][l]{149:}}%
\verb&                case collect_name(Str2) of&\\
\noindent\makebox[0pt][r]{\makebox[4.5em][l]{150:}}%
\verb&                    {ok, Tag, Rest} ->&\\
\noindent\makebox[0pt][r]{\makebox[4.5em][l]{151:}}%
\verb&                        {ok, hide, Tag, Rest};&\\
\noindent\makebox[0pt][r]{\makebox[4.5em][l]{152:}}%
\verb&                    error ->&\\
\noindent\makebox[0pt][r]{\makebox[4.5em][l]{153:}}%
\verb&                        error&\\
\noindent\makebox[0pt][r]{\makebox[4.5em][l]{154:}}%
\verb&                end;&\\
\noindent\makebox[0pt][r]{\makebox[4.5em][l]{155:}}%
\verb&        [$<,$<|Str2] ->&\\
\noindent\makebox[0pt][r]{\makebox[4.5em][l]{156:}}%
\verb&            case collect_name(Str2) of&\\
\noindent\makebox[0pt][r]{\makebox[4.5em][l]{157:}}%
\verb&                {ok, Tag, Rest} ->&\\
\noindent\makebox[0pt][r]{\makebox[4.5em][l]{158:}}%
\verb&                    {ok, show, Tag, Rest};&\\
\noindent\makebox[0pt][r]{\makebox[4.5em][l]{159:}}%
\verb&                error ->&\\
\noindent\makebox[0pt][r]{\makebox[4.5em][l]{160:}}%
\verb&                    error&\\
\noindent\makebox[0pt][r]{\makebox[4.5em][l]{161:}}%
\verb&            end;&\\
\noindent\makebox[0pt][r]{\makebox[4.5em][l]{162:}}%
\verb&        _ ->&\\
\noindent\makebox[0pt][r]{\makebox[4.5em][l]{163:}}%
\verb&            error&\\
\noindent\makebox[0pt][r]{\makebox[4.5em][l]{164:}}%
\verb&    end.&\\
\end{flushleft}

Where:

\begin{flushleft}
\label{parsing_10_13}
\noindent {\small {\sl $<<$parsing (10/13)$>>$} += $\bigtriangleup$ \pageref{parsing_9_13} \ $\bigtriangledown$ \pageref{parsing_11_13}}\\*
\tt
\noindent\makebox[0pt][r]{\makebox[4.5em][l]{165:}}%
\verb&collect_name(Str) -> collect_name(Str, []).&\\
\noindent\makebox[0pt][r]{\makebox[4.5em][l]{166:}}%
\verb&&\\
\noindent\makebox[0pt][r]{\makebox[4.5em][l]{167:}}%
\verb&collect_name([$>,$>|T], L) -> {ok, reverse(L), T};&\\
\noindent\makebox[0pt][r]{\makebox[4.5em][l]{168:}}%
\verb&collect_name([H|T], L)     -> collect_name(T, [H|L]);&\\
\noindent\makebox[0pt][r]{\makebox[4.5em][l]{169:}}%
\verb&collect_name([], _)        -> error.&\\
\end{flushleft}

    After  the tag    we   expect to   find    one  of \verb+:=+    or
\verb|+=|. This is recognised with:

\begin{flushleft}
\label{parsing_11_13}
\noindent {\small {\sl $<<$parsing (11/13)$>>$} += $\bigtriangleup$ \pageref{parsing_10_13} \ $\bigtriangledown$ \pageref{parsing_12_13}}\\*
\tt
\noindent\makebox[0pt][r]{\makebox[4.5em][l]{170:}}%
\verb&parse_after_tag(Str) ->&\\
\noindent\makebox[0pt][r]{\makebox[4.5em][l]{171:}}%
\verb&    case skip_white_space(Str) of&\\
\noindent\makebox[0pt][r]{\makebox[4.5em][l]{172:}}%
\verb&        [$+,$=|_] -> {ok, continue};&\\
\noindent\makebox[0pt][r]{\makebox[4.5em][l]{173:}}%
\verb&        [$:,$=|_] -> {ok, define};&\\
\noindent\makebox[0pt][r]{\makebox[4.5em][l]{174:}}%
\verb&        _         -> error&\\
\noindent\makebox[0pt][r]{\makebox[4.5em][l]{175:}}%
\verb&    end.&\\
\end{flushleft}


\verb+parse_data+ is called once for each line in the block.
\verb+parse_data+ assumes that any line beginning \verb+<<+ 
signals a fragment inclusion.

\begin{flushleft}
\label{parsing_12_13}
\noindent {\small {\sl $<<$parsing (12/13)$>>$} += $\bigtriangleup$ \pageref{parsing_11_13} \ $\bigtriangledown$ \pageref{parsing_13_13}}\\*
\tt
\noindent\makebox[0pt][r]{\makebox[4.5em][l]{176:}}%
\verb&parse_data({Line, Str}) ->&\\
\noindent\makebox[0pt][r]{\makebox[4.5em][l]{177:}}%
\verb&    case (Str1 = skip_white_space(Str)) of&\\
\noindent\makebox[0pt][r]{\makebox[4.5em][l]{178:}}%
\verb&        [$<,$<|T] ->&\\
\noindent\makebox[0pt][r]{\makebox[4.5em][l]{179:}}%
\verb&            case parse_tag(Str1) of&\\
\noindent\makebox[0pt][r]{\makebox[4.5em][l]{180:}}%
\verb&                {ok, show, Tag, _} ->&\\
\noindent\makebox[0pt][r]{\makebox[4.5em][l]{181:}}%
\verb&                    {Line, {include,Tag,Str}};&\\
\noindent\makebox[0pt][r]{\makebox[4.5em][l]{182:}}%
\verb&                _ ->&\\
\noindent\makebox[0pt][r]{\makebox[4.5em][l]{183:}}%
\verb&                    error({badTagRef,line,Line,"=",Str})&\\
\noindent\makebox[0pt][r]{\makebox[4.5em][l]{184:}}%
\verb&            end;&\\
\noindent\makebox[0pt][r]{\makebox[4.5em][l]{185:}}%
\verb&        _ ->&\\
\noindent\makebox[0pt][r]{\makebox[4.5em][l]{186:}}%
\verb&            {Line,Str}&\\
\noindent\makebox[0pt][r]{\makebox[4.5em][l]{187:}}%
\verb&    end.&\\
\end{flushleft}

Once the data has been parsed we number the blocks in the fragments.
For example, if there were four blocks in a fragment we would number them
1/4, 2/4, 3/4, and 4/4.  This requires two passes:

\begin{flushleft}
\label{parsing_13_13}
\noindent {\small {\sl $<<$parsing (13/13)$>>$} += $\bigtriangleup$ \pageref{parsing_12_13}}\\*
\tt
\noindent\makebox[0pt][r]{\makebox[4.5em][l]{188:}}%
\verb&number_blocks(Blocks) -> &\\
\noindent\makebox[0pt][r]{\makebox[4.5em][l]{189:}}%
\verb&    Dict0 = dict:new(),&\\
\noindent\makebox[0pt][r]{\makebox[4.5em][l]{190:}}%
\verb&    {Blocks1,Dict} = n_pass1(Blocks, [], Dict0),&\\
\noindent\makebox[0pt][r]{\makebox[4.5em][l]{191:}}%
\verb&    reverse(map(fun(X) -> n_pass2(X,Dict) end, Blocks1)).&\\
\end{flushleft}

In pass one we use a dictionary to count how many times we have seen
each block.

%2345678901234567890123456789012345678901234567890123456789
\begin{flushleft}
\label{analysis_start}
\noindent {\small {\sl $<<$analysis (1/6)$>>$} := $\bigtriangledown$ \pageref{analysis_2_6}}\\*
\tt
\noindent\makebox[0pt][r]{\makebox[4.5em][l]{192:}}%
\verb&n_pass1([{block,Line,{Hide,Tag,define},Data}|T],L,Dict) ->&\\
\noindent\makebox[0pt][r]{\makebox[4.5em][l]{193:}}%
\verb&    case dict:find(Tag, Dict) of&\\
\noindent\makebox[0pt][r]{\makebox[4.5em][l]{194:}}%
\verb&        error ->&\\
\noindent\makebox[0pt][r]{\makebox[4.5em][l]{195:}}%
\verb&            n_pass1(T,[{block2,Line,{Hide,Tag,1},Data}|L],&\\
\noindent\makebox[0pt][r]{\makebox[4.5em][l]{196:}}%
\verb&                           dict:store(Tag,1,Dict));&\\
\noindent\makebox[0pt][r]{\makebox[4.5em][l]{197:}}%
\verb&        {ok, _} ->&\\
\noindent\makebox[0pt][r]{\makebox[4.5em][l]{198:}}%
\verb&            error({multiple_def,line,Line,tag,Tag}),&\\
\noindent\makebox[0pt][r]{\makebox[4.5em][l]{199:}}%
\verb&            n_pass1(T, [error|L], Dict)&\\
\noindent\makebox[0pt][r]{\makebox[4.5em][l]{200:}}%
\verb&    end;&\\
\noindent\makebox[0pt][r]{\makebox[4.5em][l]{201:}}%
\verb&n_pass1([{block,Line,{Hide,Tag,continue},Data}|T],L,Dict) ->&\\
\noindent\makebox[0pt][r]{\makebox[4.5em][l]{202:}}%
\verb&    case dict:find(Tag, Dict) of&\\
\noindent\makebox[0pt][r]{\makebox[4.5em][l]{203:}}%
\verb&        {ok, N} ->&\\
\noindent\makebox[0pt][r]{\makebox[4.5em][l]{204:}}%
\verb&            n_pass1(T,&\\
\noindent\makebox[0pt][r]{\makebox[4.5em][l]{205:}}%
\verb&                    [{block2,Line,{Hide,Tag,N+1},Data}|L],&\\
\noindent\makebox[0pt][r]{\makebox[4.5em][l]{206:}}%
\verb&                    dict:store(Tag, N+1, Dict));&\\
\noindent\makebox[0pt][r]{\makebox[4.5em][l]{207:}}%
\verb&        error ->&\\
\noindent\makebox[0pt][r]{\makebox[4.5em][l]{208:}}%
\verb&            error({continuation_but_no_start,&\\
\noindent\makebox[0pt][r]{\makebox[4.5em][l]{209:}}%
\verb&                   line,Line,tag,Tag}),&\\
\noindent\makebox[0pt][r]{\makebox[4.5em][l]{210:}}%
\verb&            n_pass1(T, [error|L], Dict)&\\
\noindent\makebox[0pt][r]{\makebox[4.5em][l]{211:}}%
\verb&    end;&\\
\noindent\makebox[0pt][r]{\makebox[4.5em][l]{212:}}%
\verb&n_pass1([H|T], Blocks, Dict) ->&\\
\noindent\makebox[0pt][r]{\makebox[4.5em][l]{213:}}%
\verb&    n_pass1(T, [H|Blocks], Dict);&\\
\noindent\makebox[0pt][r]{\makebox[4.5em][l]{214:}}%
\verb&n_pass1([], Blocks, Dict) ->&\\
\noindent\makebox[0pt][r]{\makebox[4.5em][l]{215:}}%
\verb&    {Blocks, Dict}.&\\
\end{flushleft}

    After pass one we know the maximum number of blocks in each fragment.
In the second pass we can add this information to the block headers.

\begin{flushleft}
\label{analysis_2_6}
\noindent {\small {\sl $<<$analysis (2/6)$>>$} += $\bigtriangleup$ \pageref{analysis_start} \ $\bigtriangledown$ \pageref{analysis_3_6}}\\*
\tt
\noindent\makebox[0pt][r]{\makebox[4.5em][l]{216:}}%
\verb&n_pass2({block2, Line,{Hide,Tag,N}, Data}, Dict) ->&\\
\noindent\makebox[0pt][r]{\makebox[4.5em][l]{217:}}%
\verb&        Max = dict:fetch(Tag, Dict),&\\
\noindent\makebox[0pt][r]{\makebox[4.5em][l]{218:}}%
\verb&        {block3, Line,{Hide,Tag,{N,Max}},Data};&\\
\noindent\makebox[0pt][r]{\makebox[4.5em][l]{219:}}%
\verb&n_pass2(X, _) ->&\\
\noindent\makebox[0pt][r]{\makebox[4.5em][l]{220:}}%
\verb&    X.&\\
\end{flushleft}

  Note how in each pass we changed the tags \verb+block1+ to
\verb+block2+, then \verb+block3+ and finally \verb+block4+.  Use of
distinct tags for different phases in the processing is good
programming practise, it aids understanding of the program and in the
event of an error makes debugging easier.

    When we have numbered all the blocks we can go to the next phase
of the program which numbers all the lines {\sl within\/} the code
blocks.

\begin{flushleft}
\label{analysis_3_6}
\noindent {\small {\sl $<<$analysis (3/6)$>>$} += $\bigtriangleup$ \pageref{analysis_2_6} \ $\bigtriangledown$ \pageref{analysis_4_6}}\\*
\tt
\noindent\makebox[0pt][r]{\makebox[4.5em][l]{221:}}%
\verb&number_lines(Blocks) ->&\\
\noindent\makebox[0pt][r]{\makebox[4.5em][l]{222:}}%
\verb&    {Blocks1, {_Max, Dict}} = &\\
\noindent\makebox[0pt][r]{\makebox[4.5em][l]{223:}}%
\verb&        mapfoldl(fun number_lines/2, {1, []}, Blocks),&\\
\noindent\makebox[0pt][r]{\makebox[4.5em][l]{224:}}%
\verb&    {Blocks1, Dict}.&\\
\noindent\makebox[0pt][r]{\makebox[4.5em][l]{225:}}%
\verb&&\\
\noindent\makebox[0pt][r]{\makebox[4.5em][l]{226:}}%
\verb&number_lines({block3, Line, {show,Tag,Indx}, Data}, A) ->&\\
\noindent\makebox[0pt][r]{\makebox[4.5em][l]{227:}}%
\verb&    {Data1, A1} = mapfoldl(fun number_code/2, A, Data),&\\
\noindent\makebox[0pt][r]{\makebox[4.5em][l]{228:}}%
\verb&    {{block4, Line, {show, Tag, Indx}, Data1}, A1};&\\
\noindent\makebox[0pt][r]{\makebox[4.5em][l]{229:}}%
\verb&number_lines(Other, A) ->&\\
\noindent\makebox[0pt][r]{\makebox[4.5em][l]{230:}}%
\verb&    {Other, A}.&\\
\end{flushleft}

\verb+number_code+ numbers the individual lines of code and builds a
dictionary of any line number references which are contained 
{\sl within\/} the code blocks.

\begin{flushleft}
\label{analysis_4_6}
\noindent {\small {\sl $<<$analysis (4/6)$>>$} += $\bigtriangleup$ \pageref{analysis_3_6} \ $\bigtriangledown$ \pageref{analysis_5_6}}\\*
\tt
\noindent\makebox[0pt][r]{\makebox[4.5em][l]{231:}}%
\verb&number_code({_Line, {include,Tag,Str}},{N,Dict}) ->&\\
\noindent\makebox[0pt][r]{\makebox[4.5em][l]{232:}}%
\verb&    {{include1,N,Tag,Str},{N+1, Dict}};&\\
\noindent\makebox[0pt][r]{\makebox[4.5em][l]{233:}}%
\verb&number_code({_Line, Str}, {N, Dict}) ->&\\
\noindent\makebox[0pt][r]{\makebox[4.5em][l]{234:}}%
\verb&    case Str of&\\
\noindent\makebox[0pt][r]{\makebox[4.5em][l]{235:}}%
\verb&        [$%,$#|T] ->&\\
\noindent\makebox[0pt][r]{\makebox[4.5em][l]{236:}}%
\verb&            Label = extract_label(T),&\\
\noindent\makebox[0pt][r]{\makebox[4.5em][l]{237:}}%
\verb&            {label, {N, add_label(Label, N, Dict)}};&\\
\noindent\makebox[0pt][r]{\makebox[4.5em][l]{238:}}%
\verb&        _ ->&\\
\noindent\makebox[0pt][r]{\makebox[4.5em][l]{239:}}%
\verb&            {{line,N,Str}, {N+1,Dict}}&\\
\noindent\makebox[0pt][r]{\makebox[4.5em][l]{240:}}%
\verb&    end.&\\
\end{flushleft}

{\sl 235} -- recognises a label.         \verb+extract_label+ 
finds the value of the label.

\begin{flushleft}
\label{analysis_5_6}
\noindent {\small {\sl $<<$analysis (5/6)$>>$} += $\bigtriangleup$ \pageref{analysis_4_6} \ $\bigtriangledown$ \pageref{analysis_6_6}}\\*
\tt
\noindent\makebox[0pt][r]{\makebox[4.5em][l]{241:}}%
\verb&extract_label([$\n|_]) -> [];&\\
\noindent\makebox[0pt][r]{\makebox[4.5em][l]{242:}}%
\verb&extract_label([$\t|_]) -> [];&\\
\noindent\makebox[0pt][r]{\makebox[4.5em][l]{243:}}%
\verb&extract_label([$ |_])  -> [];&\\
\noindent\makebox[0pt][r]{\makebox[4.5em][l]{244:}}%
\verb&extract_label([])      -> [];&\\
\noindent\makebox[0pt][r]{\makebox[4.5em][l]{245:}}%
\verb&extract_label([H|T])   -> [H|extract_label(T)].&\\
\end{flushleft}

\verb+add_label+ adds a new label and line number pair to the
dictionary. It also checks for multiple lables and generates an error
if an attempt is made to define a multiple label.

\begin{flushleft}
\label{analysis_6_6}
\noindent {\small {\sl $<<$analysis (6/6)$>>$} += $\bigtriangleup$ \pageref{analysis_5_6}}\\*
\tt
\noindent\makebox[0pt][r]{\makebox[4.5em][l]{246:}}%
\verb&add_label(Label, N, Dict) ->&\\
\noindent\makebox[0pt][r]{\makebox[4.5em][l]{247:}}%
\verb&    case lists:keysearch(Label, 1, Dict) of&\\
\noindent\makebox[0pt][r]{\makebox[4.5em][l]{248:}}%
\verb&        {value, _} ->&\\
\noindent\makebox[0pt][r]{\makebox[4.5em][l]{249:}}%
\verb&            error({duplicate_label, Label}),&\\
\noindent\makebox[0pt][r]{\makebox[4.5em][l]{250:}}%
\verb&            Dict;&\\
\noindent\makebox[0pt][r]{\makebox[4.5em][l]{251:}}%
\verb&        _ ->&\\
\noindent\makebox[0pt][r]{\makebox[4.5em][l]{252:}}%
\verb&            [{Label,N}|Dict]&\\
\noindent\makebox[0pt][r]{\makebox[4.5em][l]{253:}}%
\verb&    end.&\\
\end{flushleft}

\section*{Eweave}

Eweave creates the \LaTeX{} file. \verb+OutDir+ is the directory were
the output is to go. \verb+File+ is the name of the current file.
\verb+L+ is the parsed file:

\begin{flushleft}
\label{eweave_start}
\noindent {\small {\sl $<<$eweave (1/14)$>>$} := $\bigtriangledown$ \pageref{eweave_2_14}}\\*
\tt
\noindent\makebox[0pt][r]{\makebox[4.5em][l]{254:}}%
\verb&eweave(OutDir, File, L, Dict) ->&\\
\noindent\makebox[0pt][r]{\makebox[4.5em][l]{255:}}%
\verb&    FullName = OutDir ++ "/" ++ File ++ ".tex", &\\
\noindent\makebox[0pt][r]{\makebox[4.5em][l]{256:}}%
\verb&    case file:open(FullName, write) of&\\
\noindent\makebox[0pt][r]{\makebox[4.5em][l]{257:}}%
\verb&        {ok, S} ->&\\
\noindent\makebox[0pt][r]{\makebox[4.5em][l]{258:}}%
\verb&            foreach(fun(B) -> &\\
\noindent\makebox[0pt][r]{\makebox[4.5em][l]{259:}}%
\verb&                           weave_block(B, Dict, S) &\\
\noindent\makebox[0pt][r]{\makebox[4.5em][l]{260:}}%
\verb&                   end, L),&\\
\noindent\makebox[0pt][r]{\makebox[4.5em][l]{261:}}%
\verb&            file:close(S);&\\
\noindent\makebox[0pt][r]{\makebox[4.5em][l]{262:}}%
\verb&         {error, Why } ->&\\
\noindent\makebox[0pt][r]{\makebox[4.5em][l]{263:}}%
\verb&            exit({cannot,open, FullName})&\\
\noindent\makebox[0pt][r]{\makebox[4.5em][l]{264:}}%
\verb&    end.&\\
\end{flushleft}

\verb+eweave+ calls \verb+weave_block+ for each block in the file:
Note that hidden blocks are not output:

\begin{flushleft}
\label{eweave_2_14}
\noindent {\small {\sl $<<$eweave (2/14)$>>$} += $\bigtriangleup$ \pageref{eweave_start} \ $\bigtriangledown$ \pageref{eweave_3_14}}\\*
\tt
\noindent\makebox[0pt][r]{\makebox[4.5em][l]{265:}}%
\verb&weave_block({tex, Lines}, Dict, S) ->&\\
\noindent\makebox[0pt][r]{\makebox[4.5em][l]{266:}}%
\verb&    output_tex_block(S, Dict, Lines);&\\
\noindent\makebox[0pt][r]{\makebox[4.5em][l]{267:}}%
\verb&weave_block({block4,Line,{show,Tag,{N,M}},Data},_,S) ->&\\
\noindent\makebox[0pt][r]{\makebox[4.5em][l]{268:}}%
\verb&    output_code_block(S, Tag,{N,M}, Data);&\\
\noindent\makebox[0pt][r]{\makebox[4.5em][l]{269:}}%
\verb&weave_block(_, _, _) -> &\\
\noindent\makebox[0pt][r]{\makebox[4.5em][l]{270:}}%
\verb&    true.&\\
\end{flushleft}

  \verb+weave_block+ calls \verb+output_tex_block+ or
\verb+output_code_block+ depending upon the block type.

\begin{flushleft}
\label{eweave_3_14}
\noindent {\small {\sl $<<$eweave (3/14)$>>$} += $\bigtriangleup$ \pageref{eweave_2_14} \ $\bigtriangledown$ \pageref{eweave_4_14}}\\*
\tt
\noindent\makebox[0pt][r]{\makebox[4.5em][l]{271:}}%
\verb&output_tex_block(S, Dict, Lines) ->&\\
\noindent\makebox[0pt][r]{\makebox[4.5em][l]{272:}}%
\verb&    foreach(fun({_,Str}) -> &\\
\noindent\makebox[0pt][r]{\makebox[4.5em][l]{273:}}%
\verb&                   put_tex_line(S, Dict, Str) &\\
\noindent\makebox[0pt][r]{\makebox[4.5em][l]{274:}}%
\verb&            end, Lines).&\\
\end{flushleft}

    \verb+put_text_line+ just outputs the line  if it is {\sl not\/} a
comment.

\begin{flushleft}
\label{eweave_4_14}
\noindent {\small {\sl $<<$eweave (4/14)$>>$} += $\bigtriangleup$ \pageref{eweave_3_14} \ $\bigtriangledown$ \pageref{eweave_5_14}}\\*
\tt
\noindent\makebox[0pt][r]{\makebox[4.5em][l]{275:}}%
\verb&put_tex_line(S, Dict, [$#|_]) -> true;&\\
\noindent\makebox[0pt][r]{\makebox[4.5em][l]{276:}}%
\verb&put_tex_line(S, Dict, Str)    -> &\\
\noindent\makebox[0pt][r]{\makebox[4.5em][l]{277:}}%
\verb&    Str1 = expand_labels(Str, Dict),&\\
\noindent\makebox[0pt][r]{\makebox[4.5em][l]{278:}}%
\verb&    io:fwrite(S, "~s~n", [detab(Str1)]).&\\
\end{flushleft}

\verb+expand_labels+ has the jobs of expanding any lables in the
line:

\begin{flushleft}
\label{eweave_5_14}
\noindent {\small {\sl $<<$eweave (5/14)$>>$} += $\bigtriangleup$ \pageref{eweave_4_14} \ $\bigtriangledown$ \pageref{eweave_6_14}}\\*
\tt
\noindent\makebox[0pt][r]{\makebox[4.5em][l]{279:}}%
\verb&expand_labels([$\\,$L,$N,${|T], Dict) ->&\\
\noindent\makebox[0pt][r]{\makebox[4.5em][l]{280:}}%
\verb&    {Label, T1} = get_label(T, []),&\\
\noindent\makebox[0pt][r]{\makebox[4.5em][l]{281:}}%
\verb&    Val = lookup(Label, Dict),&\\
\noindent\makebox[0pt][r]{\makebox[4.5em][l]{282:}}%
\verb&    integer_to_list(Val) ++ expand_labels(T1, Dict);&\\
\noindent\makebox[0pt][r]{\makebox[4.5em][l]{283:}}%
\verb&expand_labels([$\\,$S,$L,${|T], Dict) ->&\\
\noindent\makebox[0pt][r]{\makebox[4.5em][l]{284:}}%
\verb&    {Label, T1} = get_label(T, []),&\\
\noindent\makebox[0pt][r]{\makebox[4.5em][l]{285:}}%
\verb&    Val = lookup(Label, Dict),&\\
\noindent\makebox[0pt][r]{\makebox[4.5em][l]{286:}}%
\verb&    "{\\sl " ++ integer_to_list(Val) ++&\\
\noindent\makebox[0pt][r]{\makebox[4.5em][l]{287:}}%
\verb&        [$} | expand_labels(T1, Dict)];&\\
\noindent\makebox[0pt][r]{\makebox[4.5em][l]{288:}}%
\verb&expand_labels([H|T], Dict) ->&\\
\noindent\makebox[0pt][r]{\makebox[4.5em][l]{289:}}%
\verb&    [H|expand_labels(T, Dict)];&\\
\noindent\makebox[0pt][r]{\makebox[4.5em][l]{290:}}%
\verb&expand_labels([], Dict) ->&\\
\noindent\makebox[0pt][r]{\makebox[4.5em][l]{291:}}%
\verb&    [].&\\
\end{flushleft}

{\sl 279} -- recognises a refernce to a label somewhere within a
line of \LaTeX{} text. Label are written \verb+\L+\verb+N{XXX}+ and can
be placed anywere in the text, but they cannot be split over a line
break.

{\sl 283} -- recognises a reference to a slanted label
written \verb+\S+\verb+L{XXX}+.

\verb+get_label+ gets the label:

\begin{flushleft}
\label{eweave_6_14}
\noindent {\small {\sl $<<$eweave (6/14)$>>$} += $\bigtriangleup$ \pageref{eweave_5_14} \ $\bigtriangledown$ \pageref{eweave_7_14}}\\*
\tt
\noindent\makebox[0pt][r]{\makebox[4.5em][l]{292:}}%
\verb&get_label([$}|T], L) -> {reverse(L), T};&\\
\noindent\makebox[0pt][r]{\makebox[4.5em][l]{293:}}%
\verb&get_label([H|T],  L) -> get_label(T, [H|L]);&\\
\noindent\makebox[0pt][r]{\makebox[4.5em][l]{294:}}%
\verb&get_label([], L)     -> {reverse(L), []}.&\\
\end{flushleft}

and \verb+lookup+ finds the value of a label in the dictionary:

\begin{flushleft}
\label{eweave_7_14}
\noindent {\small {\sl $<<$eweave (7/14)$>>$} += $\bigtriangleup$ \pageref{eweave_6_14} \ $\bigtriangledown$ \pageref{eweave_8_14}}\\*
\tt
\noindent\makebox[0pt][r]{\makebox[4.5em][l]{295:}}%
\verb&lookup(Label, [{Label,Val}|_]) -> Val;&\\
\noindent\makebox[0pt][r]{\makebox[4.5em][l]{296:}}%
\verb&lookup(Label, [_|T]) -> lookup(Label, T);&\\
\noindent\makebox[0pt][r]{\makebox[4.5em][l]{297:}}%
\verb&lookup(Label, [])    -> fatal({cannot,find,label,Label}).&\\
\end{flushleft}

\verb+output_code_block+ has to generate \LaTeX{} code, similar to the
following:
\begin{verbatim}
    \verb+123:    text of line 123 ...+\\
    \verb+124:    text of line 124 ...+\\
    ...
\end{verbatim}

This is done as follows:

\begin{flushleft}
\label{eweave_8_14}
\noindent {\small {\sl $<<$eweave (8/14)$>>$} += $\bigtriangleup$ \pageref{eweave_7_14} \ $\bigtriangledown$ \pageref{eweave_9_14}}\\*
\tt
\noindent\makebox[0pt][r]{\makebox[4.5em][l]{298:}}%
\verb&output_code_block(S, Tag,{N,M}, Data) ->&\\
\noindent\makebox[0pt][r]{\makebox[4.5em][l]{299:}}%
\verb&    Label = mk_label(Tag, N, M),&\\
\noindent\makebox[0pt][r]{\makebox[4.5em][l]{300:}}%
\verb&    Refs  = mk_refs(Tag, N, M),&\\
\noindent\makebox[0pt][r]{\makebox[4.5em][l]{301:}}%
\verb&    io:fwrite(S, "\\begin{flushleft}~n"&\\
\noindent\makebox[0pt][r]{\makebox[4.5em][l]{302:}}%
\verb&                 "\\label{~s}~n", [Label]),&\\
\noindent\makebox[0pt][r]{\makebox[4.5em][l]{303:}}%
\verb&    io:fwrite(S, "\\noindent {\\small "&\\
\noindent\makebox[0pt][r]{\makebox[4.5em][l]{304:}}%
\verb&                 "{\\sl $<<$~s (~w/~w)$>>$} ~s ~s}\\\\*~n"&\\
\noindent\makebox[0pt][r]{\makebox[4.5em][l]{305:}}%
\verb&                 "\\tt~n",&\\
\noindent\makebox[0pt][r]{\makebox[4.5em][l]{306:}}%
\verb&              [str2tex(Tag),N,M,symbol(N,M),Refs]),&\\
\noindent\makebox[0pt][r]{\makebox[4.5em][l]{307:}}%
\verb&    foreach(fun(X) -> put_code(S, X) end, Data),&\\
\noindent\makebox[0pt][r]{\makebox[4.5em][l]{308:}}%
\verb&    io:fwrite(S, "\\end{flushleft}~n", []).&\\
\end{flushleft}

{\sl 303} -- {\sl 306} \verb+\obeylines+ is needed here to stop \TeX{}
complaining about "Underfull boxes". \verb+\noindent+ is to stop the
paragraph indentation which would otherwise occur here. \verb+\\*+
prevents the header from being separated from the first line of code.

{\sl 308} -- adds a blank line after every block.

\verb+put_code+ is called once for every line of code within a block.

\begin{flushleft}
\label{eweave_9_14}
\noindent {\small {\sl $<<$eweave (9/14)$>>$} += $\bigtriangleup$ \pageref{eweave_8_14} \ $\bigtriangledown$ \pageref{eweave_10_14}}\\*
\tt
\noindent\makebox[0pt][r]{\makebox[4.5em][l]{309:}}%
\verb&put_code(S, {line,N,Str}) ->&\\
\noindent\makebox[0pt][r]{\makebox[4.5em][l]{310:}}%
\verb&    Str1 = verb_prepare(Str),&\\
\noindent\makebox[0pt][r]{\makebox[4.5em][l]{311:}}%
\verb&    io:fwrite(S, "~s\\verb&\verb+&+\verb&~s&\verb+&+\verb&\\\\~n",&\\
\noindent\makebox[0pt][r]{\makebox[4.5em][l]{312:}}%
\verb&              [make_linehdr(N), detab(Str1)]);&\\
\noindent\makebox[0pt][r]{\makebox[4.5em][l]{313:}}%
\verb&put_code(S, {include1,N,Tag,Str}) ->&\\
\noindent\makebox[0pt][r]{\makebox[4.5em][l]{314:}}%
\verb&    {Blanks, _} = isolate_blanks(detab(Str)),&\\
\noindent\makebox[0pt][r]{\makebox[4.5em][l]{315:}}%
\verb&    DefRef = label(Tag,1,999999),&\\
\noindent\makebox[0pt][r]{\makebox[4.5em][l]{316:}}%
\verb&    io:fwrite(S, "~s\\verb&\verb+&+\verb&~s&\verb+&+\verb&"&\\
\noindent\makebox[0pt][r]{\makebox[4.5em][l]{317:}}%
\verb&                 "{\\sl $<<$~s \\pageref{~s} $>>$}\\\\~n",&\\
\noindent\makebox[0pt][r]{\makebox[4.5em][l]{318:}}%
\verb&              [make_linehdr(N), Blanks,&\\
\noindent\makebox[0pt][r]{\makebox[4.5em][l]{319:}}%
\verb&               detab(str2tex(Tag)), DefRef]);&\\
\noindent\makebox[0pt][r]{\makebox[4.5em][l]{320:}}%
\verb&put_code(S, label) ->&\\
\noindent\makebox[0pt][r]{\makebox[4.5em][l]{321:}}%
\verb&    void.&\\
\noindent\makebox[0pt][r]{\makebox[4.5em][l]{322:}}%
\verb&&\\
\noindent\makebox[0pt][r]{\makebox[4.5em][l]{323:}}%
\verb&make_linehdr(N) ->&\\
\noindent\makebox[0pt][r]{\makebox[4.5em][l]{324:}}%
\verb&    io_lib:fwrite("\\noindent\\makebox[0pt][r]"&\\
\noindent\makebox[0pt][r]{\makebox[4.5em][l]{325:}}%
\verb&                  "{\\makebox[4.5em][l]{~w:}}%~n", [N]).&\\
\end{flushleft}

The indentation of each fragment inclusion point is preserved, but any
text following it on the same line is lost.

The problem with outputting lines of code verbatim is that of finding a
proper delimiter. Using \verb+\begin{verbatim}+ \ldots{}
\verb+\end{verbatim}+ would make it even more difficult to assure that
the text ``\verb+\end{verbatim}+'' does not occur in the code (perhaps
in a comment or a string), so to make it simple we use \verb+\verb+ on a
line-by-line basis. We cannot guarantee that any fixed character does
not occur in the code, so we are forced to scan each line that is to be
output in order to avoid occurrences of the delimiter character. As
delimiter in the output, we select the character ``\verb+&+''. (In
particular, this is not popular for making ``bars'' like
``\verb+%%%%%%%%+'' in comment text.) Occurrences of ``\verb+&+'' in a
line are ``escaped'' by the \verb+verb_prepare+ function.

\begin{flushleft}
\label{eweave_10_14}
\noindent {\small {\sl $<<$eweave (10/14)$>>$} += $\bigtriangleup$ \pageref{eweave_9_14} \ $\bigtriangledown$ \pageref{eweave_11_14}}\\*
\tt
\noindent\makebox[0pt][r]{\makebox[4.5em][l]{326:}}%
\verb&verb_prepare([$&\verb+&+\verb& | Cs]) ->&\\
\noindent\makebox[0pt][r]{\makebox[4.5em][l]{327:}}%
\verb&    [$&\verb+&+\verb&, $\\, $v, $e, $r, $b, $+, $&\verb+&+\verb&, $+,&\\
\noindent\makebox[0pt][r]{\makebox[4.5em][l]{328:}}%
\verb&     $\\, $v, $e, $r, $b, $&\verb+&+\verb& | verb_prepare(Cs)];&\\
\noindent\makebox[0pt][r]{\makebox[4.5em][l]{329:}}%
\verb&verb_prepare([C | Cs]) ->&\\
\noindent\makebox[0pt][r]{\makebox[4.5em][l]{330:}}%
\verb&    [C | verb_prepare(Cs)];&\\
\noindent\makebox[0pt][r]{\makebox[4.5em][l]{331:}}%
\verb&verb_prepare([]) ->&\\
\noindent\makebox[0pt][r]{\makebox[4.5em][l]{332:}}%
\verb&    [].&\\
\end{flushleft}

  \verb+mk_label+ and \verb+mk_refs+ create the labels and the forward
and backward references which are used in the \LaTeX{} formatting of a
fragment. Each block in a fragment contains a reference (written
``$\bigtriangleup$ Page)'' to the previous block in the fragment and a
reference (written ``$\bigtriangledown$ Page)'' to the next block in the
current fragment. Note the special cases of the first block which has no
predecessor and the last block which has no successor.

Note also the double backslashes \verb+\\+ where a single backslash
should occur in the \LaTeX{} text.

\begin{flushleft}
\label{eweave_11_14}
\noindent {\small {\sl $<<$eweave (11/14)$>>$} += $\bigtriangleup$ \pageref{eweave_10_14} \ $\bigtriangledown$ \pageref{eweave_12_14}}\\*
\tt
\noindent\makebox[0pt][r]{\makebox[4.5em][l]{333:}}%
\verb&mk_label(Tag, N, M) ->&\\
\noindent\makebox[0pt][r]{\makebox[4.5em][l]{334:}}%
\verb&    label(Tag,N,M).&\\
\noindent\makebox[0pt][r]{\makebox[4.5em][l]{335:}}%
\verb&&\\
\noindent\makebox[0pt][r]{\makebox[4.5em][l]{336:}}%
\verb&mk_refs(Tag, 1, 1) -> "";&\\
\noindent\makebox[0pt][r]{\makebox[4.5em][l]{337:}}%
\verb&mk_refs(Tag, 1, N) when N > 1 ->&\\
\noindent\makebox[0pt][r]{\makebox[4.5em][l]{338:}}%
\verb&    io_lib:fwrite("$\\bigtriangledown$ \\pageref{~s}", &\\
\noindent\makebox[0pt][r]{\makebox[4.5em][l]{339:}}%
\verb&                  [label(Tag,2,N)]);&\\
\noindent\makebox[0pt][r]{\makebox[4.5em][l]{340:}}%
\verb&mk_refs(Tag, N, N) ->&\\
\noindent\makebox[0pt][r]{\makebox[4.5em][l]{341:}}%
\verb&    io_lib:fwrite("$\\bigtriangleup$ \\pageref{~s}", &\\
\noindent\makebox[0pt][r]{\makebox[4.5em][l]{342:}}%
\verb&                  [label(Tag,N-1,N)]);&\\
\noindent\makebox[0pt][r]{\makebox[4.5em][l]{343:}}%
\verb&mk_refs(Tag, N, M) when N < M ->&\\
\noindent\makebox[0pt][r]{\makebox[4.5em][l]{344:}}%
\verb&    io_lib:fwrite("$\\bigtriangleup$ \\pageref{~s} \\ "&\\
\noindent\makebox[0pt][r]{\makebox[4.5em][l]{345:}}%
\verb&                  "$\\bigtriangledown$ \\pageref{~s}", &\\
\noindent\makebox[0pt][r]{\makebox[4.5em][l]{346:}}%
\verb&                  [label(Tag,N-1,M),label(Tag,N+1,M)]);&\\
\noindent\makebox[0pt][r]{\makebox[4.5em][l]{347:}}%
\verb&mk_refs(_, _, _) ->&\\
\noindent\makebox[0pt][r]{\makebox[4.5em][l]{348:}}%
\verb&    "".&\\
\noindent\makebox[0pt][r]{\makebox[4.5em][l]{349:}}%
\verb&&\\
\noindent\makebox[0pt][r]{\makebox[4.5em][l]{350:}}%
\verb&label(Tag,1,_) -> Tag ++ "_start";&\\
\noindent\makebox[0pt][r]{\makebox[4.5em][l]{351:}}%
\verb&label(Tag,N,M) -> io_lib:fwrite("~s_~w_~w", [Tag,N,M]).&\\
\end{flushleft}

  Perhaps we should say something about the handling of inclusions:
these are contained in the middle of code blocks. We want to type set
the include references in italics, and correctly indent it, keeping the
indentation that was used in the source file.

  The term \verb+{include,Tag,Str}+ is the parsed form of the original
source code line; \verb+Str+ is the original line. We split off the
blanks at the start of this line and typeset the blanks as verbatim.
After this we type set the tag name in italics. This will ensure that
the inclusion directive is correctly indented.

\begin{flushleft}
\label{eweave_12_14}
\noindent {\small {\sl $<<$eweave (12/14)$>>$} += $\bigtriangleup$ \pageref{eweave_11_14} \ $\bigtriangledown$ \pageref{eweave_13_14}}\\*
\tt
\noindent\makebox[0pt][r]{\makebox[4.5em][l]{352:}}%
\verb&isolate_blanks(L) -> splitwith(fun(X) -> X == 32 end, L).&\\
\end{flushleft}

And we need to format the "type" of the block in the fragment:

\begin{flushleft}
\label{eweave_13_14}
\noindent {\small {\sl $<<$eweave (13/14)$>>$} += $\bigtriangleup$ \pageref{eweave_12_14} \ $\bigtriangledown$ \pageref{eweave_14_14}}\\*
\tt
\noindent\makebox[0pt][r]{\makebox[4.5em][l]{353:}}%
\verb&symbol(1,N) -> ":=";&\\
\noindent\makebox[0pt][r]{\makebox[4.5em][l]{354:}}%
\verb&symbol(N,M) -> "+=".&\\
\end{flushleft}

\section*{Etangle}

Etangle untangles the input 
file and creates files from all files mentioned in the
\verb+outfiles+ declaration, it {\sl returns\/} a list of all files
that it created.

\begin{flushleft}
\label{etangle_start}
\noindent {\small {\sl $<<$etangle (1/4)$>>$} := $\bigtriangledown$ \pageref{etangle_2_4}}\\*
\tt
\noindent\makebox[0pt][r]{\makebox[4.5em][l]{355:}}%
\verb&etangle(Src, OutDir, OutFiles, Copy, L) ->&\\
\noindent\makebox[0pt][r]{\makebox[4.5em][l]{356:}}%
\verb&    io:fwrite("Calling tangle~n", []),&\\
\noindent\makebox[0pt][r]{\makebox[4.5em][l]{357:}}%
\verb&    Files =  string:tokens(OutFiles, ","),&\\
\noindent\makebox[0pt][r]{\makebox[4.5em][l]{358:}}%
\verb&    AllBlocks = [{H,D} || {block4, _, H, D} <- L],&\\
\noindent\makebox[0pt][r]{\makebox[4.5em][l]{359:}}%
\verb&    UsedBlocks = &\\
\noindent\makebox[0pt][r]{\makebox[4.5em][l]{360:}}%
\verb&        case Files of&\\
\noindent\makebox[0pt][r]{\makebox[4.5em][l]{361:}}%
\verb&            [] ->&\\
\noindent\makebox[0pt][r]{\makebox[4.5em][l]{362:}}%
\verb&                io:fwrite("** Nothing to tangle~n", []),&\\
\noindent\makebox[0pt][r]{\makebox[4.5em][l]{363:}}%
\verb&                [];&\\
\noindent\makebox[0pt][r]{\makebox[4.5em][l]{364:}}%
\verb&            _ ->&\\
\noindent\makebox[0pt][r]{\makebox[4.5em][l]{365:}}%
\verb&                foldl(fun(File, Used) ->&\\
\noindent\makebox[0pt][r]{\makebox[4.5em][l]{366:}}%
\verb&                          tangle_file(Src,OutDir,File,Copy, &\\
\noindent\makebox[0pt][r]{\makebox[4.5em][l]{367:}}%
\verb&                                      AllBlocks, Used)&\\
\noindent\makebox[0pt][r]{\makebox[4.5em][l]{368:}}%
\verb&                      end, [], Files)&\\
\noindent\makebox[0pt][r]{\makebox[4.5em][l]{369:}}%
\verb&    end,&\\
\noindent\makebox[0pt][r]{\makebox[4.5em][l]{370:}}%
\verb&    warn_unused(AllBlocks, UsedBlocks),&\\
\noindent\makebox[0pt][r]{\makebox[4.5em][l]{371:}}%
\verb&    Files.&\\
\end{flushleft}

\verb+warn_unused+ prints a warning if any blocks are not used.

\begin{flushleft}
\label{etangle_2_4}
\noindent {\small {\sl $<<$etangle (2/4)$>>$} += $\bigtriangleup$ \pageref{etangle_start} \ $\bigtriangledown$ \pageref{etangle_3_4}}\\*
\tt
\noindent\makebox[0pt][r]{\makebox[4.5em][l]{372:}}%
\verb&warn_unused(All, Used) ->&\\
\noindent\makebox[0pt][r]{\makebox[4.5em][l]{373:}}%
\verb&    Defined = [{Tag,N,M} || {{_,Tag,{N,M}},_} <- All],&\\
\noindent\makebox[0pt][r]{\makebox[4.5em][l]{374:}}%
\verb&    NotUsed = filter(fun(X) -> &\\
\noindent\makebox[0pt][r]{\makebox[4.5em][l]{375:}}%
\verb&                            not member(X, Used) &\\
\noindent\makebox[0pt][r]{\makebox[4.5em][l]{376:}}%
\verb&                     end, Defined),&\\
\noindent\makebox[0pt][r]{\makebox[4.5em][l]{377:}}%
\verb&    case NotUsed of&\\
\noindent\makebox[0pt][r]{\makebox[4.5em][l]{378:}}%
\verb&        [] ->&\\
\noindent\makebox[0pt][r]{\makebox[4.5em][l]{379:}}%
\verb&            true;&\\
\noindent\makebox[0pt][r]{\makebox[4.5em][l]{380:}}%
\verb&        _ ->&\\
\noindent\makebox[0pt][r]{\makebox[4.5em][l]{381:}}%
\verb&            io:fwrite("*** warning some blocks unused..~n"&\\
\noindent\makebox[0pt][r]{\makebox[4.5em][l]{382:}}%
\verb&                      "~p~n", [NotUsed])&\\
\noindent\makebox[0pt][r]{\makebox[4.5em][l]{383:}}%
\verb&    end.&\\
\end{flushleft}
    \verb+tangle_file+ is   called   for each file  mentioned   in the
\verb+outfiles+ declaration. The variable  \verb+Used+ is used to
note all fragments which have been included from the original source file.
    
If the file name starts with a \verb|+| character the file
mode is set to executable.

\begin{flushleft}
\label{etangle_3_4}
\noindent {\small {\sl $<<$etangle (3/4)$>>$} += $\bigtriangleup$ \pageref{etangle_2_4} \ $\bigtriangledown$ \pageref{etangle_4_4}}\\*
\tt
\noindent\makebox[0pt][r]{\makebox[4.5em][l]{384:}}%
\verb&tangle_file(Src, OutDir, File, Copy, Blocks, Used) ->&\\
\noindent\makebox[0pt][r]{\makebox[4.5em][l]{385:}}%
\verb&    BaseName = case File of &\\
\noindent\makebox[0pt][r]{\makebox[4.5em][l]{386:}}%
\verb&                   [$+|T] -> T;&\\
\noindent\makebox[0pt][r]{\makebox[4.5em][l]{387:}}%
\verb&                   T      -> T&\\
\noindent\makebox[0pt][r]{\makebox[4.5em][l]{388:}}%
\verb&               end,&\\
\noindent\makebox[0pt][r]{\makebox[4.5em][l]{389:}}%
\verb&    FullName = OutDir ++ "/" ++ BaseName,&\\
\noindent\makebox[0pt][r]{\makebox[4.5em][l]{390:}}%
\verb&    io:fwrite("Creating ~s~n", [FullName]),&\\
\noindent\makebox[0pt][r]{\makebox[4.5em][l]{391:}}%
\verb&    {ok, S} = file:open(FullName, write),&\\
\noindent\makebox[0pt][r]{\makebox[4.5em][l]{392:}}%
\verb&    case lists:suffix(".erl", File) of&\\
\noindent\makebox[0pt][r]{\makebox[4.5em][l]{393:}}%
\verb&        true ->&\\
\noindent\makebox[0pt][r]{\makebox[4.5em][l]{394:}}%
\verb&            Warn = "%% This file was created from ~s~n"&\\
\noindent\makebox[0pt][r]{\makebox[4.5em][l]{395:}}%
\verb&                   "%% please do not edit ~n"&\\
\noindent\makebox[0pt][r]{\makebox[4.5em][l]{396:}}%
\verb&                   "%% ~s~n",&\\
\noindent\makebox[0pt][r]{\makebox[4.5em][l]{397:}}%
\verb&            io:fwrite(S, Warn, [Src,Copy]);&\\
\noindent\makebox[0pt][r]{\makebox[4.5em][l]{398:}}%
\verb&        false ->&\\
\noindent\makebox[0pt][r]{\makebox[4.5em][l]{399:}}%
\verb&            true&\\
\noindent\makebox[0pt][r]{\makebox[4.5em][l]{400:}}%
\verb&    end,&\\
\noindent\makebox[0pt][r]{\makebox[4.5em][l]{401:}}%
\verb&    Used1 = include_file(S, {start, BaseName}, &\\
\noindent\makebox[0pt][r]{\makebox[4.5em][l]{402:}}%
\verb&                         Blocks, [], Used),&\\
\noindent\makebox[0pt][r]{\makebox[4.5em][l]{403:}}%
\verb&    file:close(S),&\\
\noindent\makebox[0pt][r]{\makebox[4.5em][l]{404:}}%
\verb&    case File of&\\
\noindent\makebox[0pt][r]{\makebox[4.5em][l]{405:}}%
\verb&        [$+|_] ->&\\
\noindent\makebox[0pt][r]{\makebox[4.5em][l]{406:}}%
\verb&            os:cmd("chmod u+x " ++ BaseName);&\\
\noindent\makebox[0pt][r]{\makebox[4.5em][l]{407:}}%
\verb&        _ ->&\\
\noindent\makebox[0pt][r]{\makebox[4.5em][l]{408:}}%
\verb&            true&\\
\noindent\makebox[0pt][r]{\makebox[4.5em][l]{409:}}%
\verb&    end,&\\
\noindent\makebox[0pt][r]{\makebox[4.5em][l]{410:}}%
\verb&    Used1.&\\
\end{flushleft}

    Most  of    the   work  is    done    in  \verb+include_file+  and
\verb+untangle+. The  only point to  note  is the  use of the variable
\verb+Path+.  Path is a list  of all references  from  the top of  the
inclusion  tree to the current node  in the tree.  We keep a record of
how  we have got  to   the current node   in  order to  avoid  circular
inclusions. A circular  inclusion  (i.e. one of  the form  A includes B
includes C ... includes A) would put the system into an infinite loop.

  A  fatal  error is  generated   if we attempt   to  make a  circular
inclusion.  We also keep track (in  the variable \verb+Used+) of which
blocks in  the fragment have been included.    This infomation will be
used later so that we can we  can issue a  diagnostic if any fragments
were missed.

\begin{flushleft}
\label{etangle_4_4}
\noindent {\small {\sl $<<$etangle (4/4)$>>$} += $\bigtriangleup$ \pageref{etangle_3_4}}\\*
\tt
\noindent\makebox[0pt][r]{\makebox[4.5em][l]{411:}}%
\verb&include_file(S, Tag, Blocks, Path, Used) ->&\\
\noindent\makebox[0pt][r]{\makebox[4.5em][l]{412:}}%
\verb&    {Tag1, Data}  = find_block(Tag, Blocks),&\\
\noindent\makebox[0pt][r]{\makebox[4.5em][l]{413:}}%
\verb&    case member(Tag1, Path) of&\\
\noindent\makebox[0pt][r]{\makebox[4.5em][l]{414:}}%
\verb&        true ->&\\
\noindent\makebox[0pt][r]{\makebox[4.5em][l]{415:}}%
\verb&            fatal({recursive_tangle, Tag, Path});&\\
\noindent\makebox[0pt][r]{\makebox[4.5em][l]{416:}}%
\verb&        false ->&\\
\noindent\makebox[0pt][r]{\makebox[4.5em][l]{417:}}%
\verb&            untangle(S, Data, Tag1, Blocks,&\\
\noindent\makebox[0pt][r]{\makebox[4.5em][l]{418:}}%
\verb&                     [Tag1|Path], [Tag1|Used])&\\
\noindent\makebox[0pt][r]{\makebox[4.5em][l]{419:}}%
\verb&    end.&\\
\noindent\makebox[0pt][r]{\makebox[4.5em][l]{420:}}%
\verb&&\\
\noindent\makebox[0pt][r]{\makebox[4.5em][l]{421:}}%
\verb&untangle(S,[{line,_,Str}|T], Tag,Blocks,Path,Used) ->&\\
\noindent\makebox[0pt][r]{\makebox[4.5em][l]{422:}}%
\verb&    io:fwrite(S, "~s~n", [Str]),&\\
\noindent\makebox[0pt][r]{\makebox[4.5em][l]{423:}}%
\verb&    untangle(S, T, Tag, Blocks, Path, Used);&\\
\noindent\makebox[0pt][r]{\makebox[4.5em][l]{424:}}%
\verb&untangle(S,[{include1,_,NextTag,_}|T],&\\
\noindent\makebox[0pt][r]{\makebox[4.5em][l]{425:}}%
\verb&         Tag,Blocks,Path,Used) ->&\\
\noindent\makebox[0pt][r]{\makebox[4.5em][l]{426:}}%
\verb&    Used1 = include_file(S, {start, NextTag},&\\
\noindent\makebox[0pt][r]{\makebox[4.5em][l]{427:}}%
\verb&                         Blocks,Path,Used),&\\
\noindent\makebox[0pt][r]{\makebox[4.5em][l]{428:}}%
\verb&    untangle(S, T, Tag, Blocks, Path, Used1);&\\
\noindent\makebox[0pt][r]{\makebox[4.5em][l]{429:}}%
\verb&untangle(S, [label|T], Tag,Blocks,Path,Used) ->&\\
\noindent\makebox[0pt][r]{\makebox[4.5em][l]{430:}}%
\verb&    untangle(S, T, Tag, Blocks, Path, Used);&\\
\noindent\makebox[0pt][r]{\makebox[4.5em][l]{431:}}%
\verb&untangle(S, [], {Tag, Max, Max}, Blocks, Path, Used) ->&\\
\noindent\makebox[0pt][r]{\makebox[4.5em][l]{432:}}%
\verb&    Used;&\\
\noindent\makebox[0pt][r]{\makebox[4.5em][l]{433:}}%
\verb&untangle(S, [], {Tag, N, Max},&\\
\noindent\makebox[0pt][r]{\makebox[4.5em][l]{434:}}%
\verb&         Blocks, Path, Used) ->&\\
\noindent\makebox[0pt][r]{\makebox[4.5em][l]{435:}}%
\verb&    include_file(S, {Tag,N+1,Max}, Blocks, Path, Used).&\\
\end{flushleft}

\verb+find_block+ finds a given block in a fragment:

\begin{flushleft}
\label{eweave_14_14}
\noindent {\small {\sl $<<$eweave (14/14)$>>$} += $\bigtriangleup$ \pageref{eweave_13_14}}\\*
\tt
\noindent\makebox[0pt][r]{\makebox[4.5em][l]{436:}}%
\verb&find_block({start, Tag}, Blocks) ->&\\
\noindent\makebox[0pt][r]{\makebox[4.5em][l]{437:}}%
\verb&    find_block(Tag, 1, Blocks);&\\
\noindent\makebox[0pt][r]{\makebox[4.5em][l]{438:}}%
\verb&find_block({Tag,I,N}, Blocks) ->&\\
\noindent\makebox[0pt][r]{\makebox[4.5em][l]{439:}}%
\verb&    find_block(Tag, I, Blocks).&\\
\noindent\makebox[0pt][r]{\makebox[4.5em][l]{440:}}%
\verb&&\\
\noindent\makebox[0pt][r]{\makebox[4.5em][l]{441:}}%
\verb&find_block(Tag, I, [{{_Hide,Tag,{I,N}},Data}|_]) ->&\\
\noindent\makebox[0pt][r]{\makebox[4.5em][l]{442:}}%
\verb&    {{Tag, I, N}, Data};&\\
\noindent\makebox[0pt][r]{\makebox[4.5em][l]{443:}}%
\verb&find_block(Tag, I, [H|Blocks]) ->&\\
\noindent\makebox[0pt][r]{\makebox[4.5em][l]{444:}}%
\verb&    find_block(Tag, I, Blocks);&\\
\noindent\makebox[0pt][r]{\makebox[4.5em][l]{445:}}%
\verb&find_block(Tag, I, []) ->&\\
\noindent\makebox[0pt][r]{\makebox[4.5em][l]{446:}}%
\verb&    fatal({cannot_find_block, I, Tag}).&\\
\end{flushleft}

\verb+file2strings+, which we mentioned earlier, is implemented using\\
\verb+binary2strings+ (this method is, incidentally, some three times
faster than a previous version which I had written using
\verb+file:read+ and \verb+io:get_line+).

\begin{flushleft}
\label{utilities_3_8}
\noindent {\small {\sl $<<$utilities (3/8)$>>$} += $\bigtriangleup$ \pageref{utilities_2_8} \ $\bigtriangledown$ \pageref{utilities_4_8}}\\*
\tt
\noindent\makebox[0pt][r]{\makebox[4.5em][l]{447:}}%
\verb&binary2strings(Bin) when binary(Bin) ->&\\
\noindent\makebox[0pt][r]{\makebox[4.5em][l]{448:}}%
\verb&    binary2strings(binary_to_list(Bin), 1, [], []).&\\
\noindent\makebox[0pt][r]{\makebox[4.5em][l]{449:}}%
\verb&&\\
\noindent\makebox[0pt][r]{\makebox[4.5em][l]{450:}}%
\verb&binary2strings([H|T], N, C, L) ->&\\
\noindent\makebox[0pt][r]{\makebox[4.5em][l]{451:}}%
\verb&    case H of&\\
\noindent\makebox[0pt][r]{\makebox[4.5em][l]{452:}}%
\verb&        $\n -> &\\
\noindent\makebox[0pt][r]{\makebox[4.5em][l]{453:}}%
\verb&            binary2strings(T,N+1,[],[{N,reverse(C)}|L]);&\\
\noindent\makebox[0pt][r]{\makebox[4.5em][l]{454:}}%
\verb&        _ -> &\\
\noindent\makebox[0pt][r]{\makebox[4.5em][l]{455:}}%
\verb&            binary2strings(T,N,[H|C],L)&\\
\noindent\makebox[0pt][r]{\makebox[4.5em][l]{456:}}%
\verb&    end;&\\
\noindent\makebox[0pt][r]{\makebox[4.5em][l]{457:}}%
\verb&binary2strings([], N, [], L) ->&\\
\noindent\makebox[0pt][r]{\makebox[4.5em][l]{458:}}%
\verb&    reverse(L);&\\
\noindent\makebox[0pt][r]{\makebox[4.5em][l]{459:}}%
\verb&binary2strings([], N, C, L) ->&\\
\noindent\makebox[0pt][r]{\makebox[4.5em][l]{460:}}%
\verb&    reverse([{N,reverse(C)}|L]).&\\
\end{flushleft}

  We also need an import declarations and a  few small test and
utility routines.


\begin{flushleft}
\label{imports_start}
\noindent {\small {\sl $<<$imports (1/1)$>>$} := }\\*
\tt
\noindent\makebox[0pt][r]{\makebox[4.5em][l]{461:}}%
\verb&-import(lists, [dropwhile/2, filter/2, foldl/3, foreach/2, &\\
\noindent\makebox[0pt][r]{\makebox[4.5em][l]{462:}}%
\verb&                map/2, mapfoldl/3, member/2, reverse/1, &\\
\noindent\makebox[0pt][r]{\makebox[4.5em][l]{463:}}%
\verb&                splitwith/2, sublist/3, suffix/2]).&\\
\end{flushleft}

\begin{flushleft}
\label{utilities_4_8}
\noindent {\small {\sl $<<$utilities (4/8)$>>$} += $\bigtriangleup$ \pageref{utilities_3_8} \ $\bigtriangledown$ \pageref{utilities_5_8}}\\*
\tt
\noindent\makebox[0pt][r]{\makebox[4.5em][l]{464:}}%
\verb&skip_white_space(Str) ->&\\
\noindent\makebox[0pt][r]{\makebox[4.5em][l]{465:}}%
\verb&    dropwhile(fun($ )  -> true;&\\
\noindent\makebox[0pt][r]{\makebox[4.5em][l]{466:}}%
\verb&                 ($\n) -> true;&\\
\noindent\makebox[0pt][r]{\makebox[4.5em][l]{467:}}%
\verb&                 ($\t) -> true;&\\
\noindent\makebox[0pt][r]{\makebox[4.5em][l]{468:}}%
\verb&                 (_)   -> false &\\
\noindent\makebox[0pt][r]{\makebox[4.5em][l]{469:}}%
\verb&              end, Str).&\\
\noindent\makebox[0pt][r]{\makebox[4.5em][l]{470:}}%
\verb&                  &\\
\noindent\makebox[0pt][r]{\makebox[4.5em][l]{471:}}%
\verb&detab(L) -> detab(L, 1, []).&\\
\noindent\makebox[0pt][r]{\makebox[4.5em][l]{472:}}%
\verb&&\\
\noindent\makebox[0pt][r]{\makebox[4.5em][l]{473:}}%
\verb&detab([$\t|T], N, L) when N rem 8 == 0 -> &\\
\noindent\makebox[0pt][r]{\makebox[4.5em][l]{474:}}%
\verb&    detab(T, N+1, [$ |L]);&\\
\noindent\makebox[0pt][r]{\makebox[4.5em][l]{475:}}%
\verb&detab([$\t|T], N, L) -> &\\
\noindent\makebox[0pt][r]{\makebox[4.5em][l]{476:}}%
\verb&    detab([$\t|T], N+1, [$ |L]);&\\
\noindent\makebox[0pt][r]{\makebox[4.5em][l]{477:}}%
\verb&detab([H|T], N, L) ->&\\
\noindent\makebox[0pt][r]{\makebox[4.5em][l]{478:}}%
\verb&    detab(T, N+1, [H|L]);&\\
\noindent\makebox[0pt][r]{\makebox[4.5em][l]{479:}}%
\verb&detab([], _, L) ->&\\
\noindent\makebox[0pt][r]{\makebox[4.5em][l]{480:}}%
\verb&    reverse(L).&\\
\noindent\makebox[0pt][r]{\makebox[4.5em][l]{481:}}%
\verb&&\\
\end{flushleft}

And a shell script to launch everything: 

\begin{flushleft}
\label{eweb_start}
\noindent {\small {\sl $<<$eweb (1/1)$>>$} := }\\*
\tt
\noindent\makebox[0pt][r]{\makebox[4.5em][l]{482:}}%
\verb&erl -noshell -pa "/home/joe/erl/eweb" \&\\
\noindent\makebox[0pt][r]{\makebox[4.5em][l]{483:}}%
\verb&    -s eweb dir -s erlang halt&\\
\end{flushleft}

\noindent (Edit the resulting file to point to where you store your
compiled version of the \verb+eweb.erl+ file.)

The shell script calls \verb+eweb:dir+ which processes all out of date
files:

\begin{flushleft}
\label{utilities_5_8}
\noindent {\small {\sl $<<$utilities (5/8)$>>$} += $\bigtriangleup$ \pageref{utilities_4_8} \ $\bigtriangledown$ \pageref{utilities_6_8}}\\*
\tt
\noindent\makebox[0pt][r]{\makebox[4.5em][l]{484:}}%
\verb&dir() -> &\\
\noindent\makebox[0pt][r]{\makebox[4.5em][l]{485:}}%
\verb&    foreach(fun file/1, &\\
\noindent\makebox[0pt][r]{\makebox[4.5em][l]{486:}}%
\verb&            find_out_of_date(".", ".lit", ".tex")).&\\
\end{flushleft}

    \verb+find_out_of_date(".", ".lit", ".tex")+ finds all \verb+.tex+
files that are out of date with respect to the \verb+.lit+ files in the
current directory:

\begin{flushleft}
\label{utilities_6_8}
\noindent {\small {\sl $<<$utilities (6/8)$>>$} += $\bigtriangleup$ \pageref{utilities_5_8} \ $\bigtriangledown$ \pageref{utilities_7_8}}\\*
\tt
\noindent\makebox[0pt][r]{\makebox[4.5em][l]{487:}}%
\verb&find_out_of_date(Dir, In, Out) ->&\\
\noindent\makebox[0pt][r]{\makebox[4.5em][l]{488:}}%
\verb&    case file:list_dir(Dir) of&\\
\noindent\makebox[0pt][r]{\makebox[4.5em][l]{489:}}%
\verb&        {ok, Files0} ->&\\
\noindent\makebox[0pt][r]{\makebox[4.5em][l]{490:}}%
\verb&            Files1 = filter(fun(F) -> &\\
\noindent\makebox[0pt][r]{\makebox[4.5em][l]{491:}}%
\verb&                                   suffix(In, F) &\\
\noindent\makebox[0pt][r]{\makebox[4.5em][l]{492:}}%
\verb&                            end, Files0),&\\
\noindent\makebox[0pt][r]{\makebox[4.5em][l]{493:}}%
\verb&            Files2 = map(fun(F) -> &\\
\noindent\makebox[0pt][r]{\makebox[4.5em][l]{494:}}%
\verb&                           sublist(F,1,&\\
\noindent\makebox[0pt][r]{\makebox[4.5em][l]{495:}}%
\verb&                                   length(F) - length(In)) &\\
\noindent\makebox[0pt][r]{\makebox[4.5em][l]{496:}}%
\verb&                         end, Files1),&\\
\noindent\makebox[0pt][r]{\makebox[4.5em][l]{497:}}%
\verb&            filter(fun(F) ->&\\
\noindent\makebox[0pt][r]{\makebox[4.5em][l]{498:}}%
\verb&                          update(F, In, Out) &\\
\noindent\makebox[0pt][r]{\makebox[4.5em][l]{499:}}%
\verb&                   end, Files2);&\\
\noindent\makebox[0pt][r]{\makebox[4.5em][l]{500:}}%
\verb&        _ ->&\\
\noindent\makebox[0pt][r]{\makebox[4.5em][l]{501:}}%
\verb&            []&\\
\noindent\makebox[0pt][r]{\makebox[4.5em][l]{502:}}%
\verb&    end.&\\
\end{flushleft}

This calls \verb+update(File, In, Out)+ which 
returns \verb+true+ if \verb|File ++ In| 
was modified after \verb|File ++ Out|.

\begin{flushleft}
\label{utilities_7_8}
\noindent {\small {\sl $<<$utilities (7/8)$>>$} += $\bigtriangleup$ \pageref{utilities_6_8} \ $\bigtriangledown$ \pageref{utilities_8_8}}\\*
\tt
\noindent\makebox[0pt][r]{\makebox[4.5em][l]{503:}}%
\verb&update(File, In, Out) ->&\\
\noindent\makebox[0pt][r]{\makebox[4.5em][l]{504:}}%
\verb&    InFile  = File ++ In,&\\
\noindent\makebox[0pt][r]{\makebox[4.5em][l]{505:}}%
\verb&    OutFile = File ++ Out,&\\
\noindent\makebox[0pt][r]{\makebox[4.5em][l]{506:}}%
\verb&    case is_file(OutFile) of&\\
\noindent\makebox[0pt][r]{\makebox[4.5em][l]{507:}}%
\verb&        true ->&\\
\noindent\makebox[0pt][r]{\makebox[4.5em][l]{508:}}%
\verb&            case writeable(OutFile) of&\\
\noindent\makebox[0pt][r]{\makebox[4.5em][l]{509:}}%
\verb&                true ->&\\
\noindent\makebox[0pt][r]{\makebox[4.5em][l]{510:}}%
\verb&                    outofdate(InFile, OutFile);&\\
\noindent\makebox[0pt][r]{\makebox[4.5em][l]{511:}}%
\verb&                false ->&\\
\noindent\makebox[0pt][r]{\makebox[4.5em][l]{512:}}%
\verb&                    %% can't write so we can't update&\\
\noindent\makebox[0pt][r]{\makebox[4.5em][l]{513:}}%
\verb&                    false&\\
\noindent\makebox[0pt][r]{\makebox[4.5em][l]{514:}}%
\verb&            end;&\\
\noindent\makebox[0pt][r]{\makebox[4.5em][l]{515:}}%
\verb&        false ->&\\
\noindent\makebox[0pt][r]{\makebox[4.5em][l]{516:}}%
\verb&            %% doesn't exist&\\
\noindent\makebox[0pt][r]{\makebox[4.5em][l]{517:}}%
\verb&            true&\\
\noindent\makebox[0pt][r]{\makebox[4.5em][l]{518:}}%
\verb&    end.&\\
\end{flushleft}

  And finally, there are number of  small utilities which interface us
to the file system:

\begin{flushleft}
\label{utilities_8_8}
\noindent {\small {\sl $<<$utilities (8/8)$>>$} += $\bigtriangleup$ \pageref{utilities_7_8}}\\*
\tt
\noindent\makebox[0pt][r]{\makebox[4.5em][l]{519:}}%
\verb&is_file(File) ->&\\
\noindent\makebox[0pt][r]{\makebox[4.5em][l]{520:}}%
\verb&    case file:file_info(File) of&\\
\noindent\makebox[0pt][r]{\makebox[4.5em][l]{521:}}%
\verb&        {ok, _} ->&\\
\noindent\makebox[0pt][r]{\makebox[4.5em][l]{522:}}%
\verb&            true;&\\
\noindent\makebox[0pt][r]{\makebox[4.5em][l]{523:}}%
\verb&        _ ->&\\
\noindent\makebox[0pt][r]{\makebox[4.5em][l]{524:}}%
\verb&            false&\\
\noindent\makebox[0pt][r]{\makebox[4.5em][l]{525:}}%
\verb&    end.&\\
\noindent\makebox[0pt][r]{\makebox[4.5em][l]{526:}}%
\verb&&\\
\noindent\makebox[0pt][r]{\makebox[4.5em][l]{527:}}%
\verb&writeable(F) ->&\\
\noindent\makebox[0pt][r]{\makebox[4.5em][l]{528:}}%
\verb&    case file:file_info(F) of&\\
\noindent\makebox[0pt][r]{\makebox[4.5em][l]{529:}}%
\verb&        {ok, {_,_,read_write,_,_,_,_}} -> true;&\\
\noindent\makebox[0pt][r]{\makebox[4.5em][l]{530:}}%
\verb&        {ok, {_,_,write     ,_,_,_,_}} -> true;&\\
\noindent\makebox[0pt][r]{\makebox[4.5em][l]{531:}}%
\verb&        _ -> false&\\
\noindent\makebox[0pt][r]{\makebox[4.5em][l]{532:}}%
\verb&    end.&\\
\noindent\makebox[0pt][r]{\makebox[4.5em][l]{533:}}%
\verb&&\\
\noindent\makebox[0pt][r]{\makebox[4.5em][l]{534:}}%
\verb&outofdate(In, Out) ->&\\
\noindent\makebox[0pt][r]{\makebox[4.5em][l]{535:}}%
\verb&    case {last_modified(In), last_modified(Out)} of&\\
\noindent\makebox[0pt][r]{\makebox[4.5em][l]{536:}}%
\verb&        {T1, T2} when T1 > T2 ->&\\
\noindent\makebox[0pt][r]{\makebox[4.5em][l]{537:}}%
\verb&            true;&\\
\noindent\makebox[0pt][r]{\makebox[4.5em][l]{538:}}%
\verb&        _ ->&\\
\noindent\makebox[0pt][r]{\makebox[4.5em][l]{539:}}%
\verb&            false&\\
\noindent\makebox[0pt][r]{\makebox[4.5em][l]{540:}}%
\verb&    end.&\\
\noindent\makebox[0pt][r]{\makebox[4.5em][l]{541:}}%
\verb&&\\
\noindent\makebox[0pt][r]{\makebox[4.5em][l]{542:}}%
\verb&last_modified(F) ->&\\
\noindent\makebox[0pt][r]{\makebox[4.5em][l]{543:}}%
\verb&    case file:file_info(F) of&\\
\noindent\makebox[0pt][r]{\makebox[4.5em][l]{544:}}%
\verb&        {ok, {_, _, _, _, Time, _, _}} ->&\\
\noindent\makebox[0pt][r]{\makebox[4.5em][l]{545:}}%
\verb&            Time;&\\
\noindent\makebox[0pt][r]{\makebox[4.5em][l]{546:}}%
\verb&        _ ->&\\
\noindent\makebox[0pt][r]{\makebox[4.5em][l]{547:}}%
\verb&            exit({last_modified, F})&\\
\noindent\makebox[0pt][r]{\makebox[4.5em][l]{548:}}%
\verb&    end.&\\
\end{flushleft}

That's it!

\section*{Utilities}

Here are a few useful utilites

\begin{flushleft}
\label{Makefile_start}
\noindent {\small {\sl $<<$Makefile (1/1)$>>$} := }\\*
\tt
\noindent\makebox[0pt][r]{\makebox[4.5em][l]{549:}}%
\verb&# Unix-Makefile for `eweb.lit'&\\
\noindent\makebox[0pt][r]{\makebox[4.5em][l]{550:}}%
\verb&#&\\
\noindent\makebox[0pt][r]{\makebox[4.5em][l]{551:}}%
\verb&# Note that `eweb.beam' must already exist for the `eweb'&\\
\noindent\makebox[0pt][r]{\makebox[4.5em][l]{552:}}%
\verb&# shell script to be able to execute EWEB in Erlang&\\
\noindent\makebox[0pt][r]{\makebox[4.5em][l]{553:}}%
\verb&# (assuming we run under BEAM).&\\
\noindent\makebox[0pt][r]{\makebox[4.5em][l]{554:}}%
\verb&#&\\
\noindent\makebox[0pt][r]{\makebox[4.5em][l]{555:}}%
\verb&# Be sure to have a backup of the latest stable `eweb.erl'&\\
\noindent\makebox[0pt][r]{\makebox[4.5em][l]{556:}}%
\verb&# (or `eweb.beam') file before running `make'!&\\
\noindent\makebox[0pt][r]{\makebox[4.5em][l]{557:}}%
\verb&# If the backup of `eweb.erl' is named `eweb.erl.stable',&\\
\noindent\makebox[0pt][r]{\makebox[4.5em][l]{558:}}%
\verb&# then just do `make restore' to get a working `eweb.beam'.&\\
\noindent\makebox[0pt][r]{\makebox[4.5em][l]{559:}}%
\verb&#&\\
\noindent\makebox[0pt][r]{\makebox[4.5em][l]{560:}}%
\verb&# The version of LaTeX used for typesetting `eweb.tex'&\\
\noindent\makebox[0pt][r]{\makebox[4.5em][l]{561:}}%
\verb&# should preferably be 2.09, rather than LaTeX 2e; the&\\
\noindent\makebox[0pt][r]{\makebox[4.5em][l]{562:}}%
\verb&# old version is generally named `latex209' if it has&\\
\noindent\makebox[0pt][r]{\makebox[4.5em][l]{563:}}%
\verb&# been kept on your local system.&\\
\noindent\makebox[0pt][r]{\makebox[4.5em][l]{564:}}%
\verb&&\\
\noindent\makebox[0pt][r]{\makebox[4.5em][l]{565:}}%
\verb&#LATEX=latex209&\\
\noindent\makebox[0pt][r]{\makebox[4.5em][l]{566:}}%
\verb&LATEX=latex&\\
\noindent\makebox[0pt][r]{\makebox[4.5em][l]{567:}}%
\verb&&\\
\noindent\makebox[0pt][r]{\makebox[4.5em][l]{568:}}%
\verb&STABLE=eweb.erl.stable&\\
\noindent\makebox[0pt][r]{\makebox[4.5em][l]{569:}}%
\verb&MACHINE=beam&\\
\noindent\makebox[0pt][r]{\makebox[4.5em][l]{570:}}%
\verb&&\\
\noindent\makebox[0pt][r]{\makebox[4.5em][l]{571:}}%
\verb&&\\
\noindent\makebox[0pt][r]{\makebox[4.5em][l]{572:}}%
\verb&all:    eweb.$(MACHINE) doc&\\
\noindent\makebox[0pt][r]{\makebox[4.5em][l]{573:}}%
\verb&&\\
\noindent\makebox[0pt][r]{\makebox[4.5em][l]{574:}}%
\verb&doc:    eweb.dvi&\\
\noindent\makebox[0pt][r]{\makebox[4.5em][l]{575:}}%
\verb&&\\
\noindent\makebox[0pt][r]{\makebox[4.5em][l]{576:}}%
\verb&ps:     eweb.ps&\\
\noindent\makebox[0pt][r]{\makebox[4.5em][l]{577:}}%
\verb&&\\
\noindent\makebox[0pt][r]{\makebox[4.5em][l]{578:}}%
\verb&restore:        clean&\\
\noindent\makebox[0pt][r]{\makebox[4.5em][l]{579:}}%
\verb&        cp -pf $(STABLE) eweb.erl&\\
\noindent\makebox[0pt][r]{\makebox[4.5em][l]{580:}}%
\verb&        erl -compile eweb&\\
\noindent\makebox[0pt][r]{\makebox[4.5em][l]{581:}}%
\verb&        touch -r $(STABLE) eweb.$(MACHINE)&\\
\noindent\makebox[0pt][r]{\makebox[4.5em][l]{582:}}%
\verb&&\\
\noindent\makebox[0pt][r]{\makebox[4.5em][l]{583:}}%
\verb&eweb.$(MACHINE):        eweb.erl&\\
\noindent\makebox[0pt][r]{\makebox[4.5em][l]{584:}}%
\verb&        erl -compile eweb&\\
\noindent\makebox[0pt][r]{\makebox[4.5em][l]{585:}}%
\verb&&\\
\noindent\makebox[0pt][r]{\makebox[4.5em][l]{586:}}%
\verb&eweb.erl eweb.tex:      eweb.lit&\\
\noindent\makebox[0pt][r]{\makebox[4.5em][l]{587:}}%
\verb&        sh ./eweb&\\
\noindent\makebox[0pt][r]{\makebox[4.5em][l]{588:}}%
\verb&&\\
\noindent\makebox[0pt][r]{\makebox[4.5em][l]{589:}}%
\verb&eweb.dvi:       eweb.tex&\\
\noindent\makebox[0pt][r]{\makebox[4.5em][l]{590:}}%
\verb&        $(LATEX) eweb.tex&\\
\noindent\makebox[0pt][r]{\makebox[4.5em][l]{591:}}%
\verb&        $(LATEX) eweb.tex&\\
\noindent\makebox[0pt][r]{\makebox[4.5em][l]{592:}}%
\verb&&\\
\noindent\makebox[0pt][r]{\makebox[4.5em][l]{593:}}%
\verb&eweb.ps:        eweb.dvi&\\
\noindent\makebox[0pt][r]{\makebox[4.5em][l]{594:}}%
\verb&        dvips eweb.dvi -o&\\
\noindent\makebox[0pt][r]{\makebox[4.5em][l]{595:}}%
\verb&&\\
\noindent\makebox[0pt][r]{\makebox[4.5em][l]{596:}}%
\verb&clean:&\\
\noindent\makebox[0pt][r]{\makebox[4.5em][l]{597:}}%
\verb&        -rm eweb.tex eweb.log eweb.aux eweb.erl&\\
\noindent\makebox[0pt][r]{\makebox[4.5em][l]{598:}}%
\verb&&\\
\noindent\makebox[0pt][r]{\makebox[4.5em][l]{599:}}%
\verb&realclean:      clean&\\
\noindent\makebox[0pt][r]{\makebox[4.5em][l]{600:}}%
\verb&        -rm eweb.dvi&\\
\end{flushleft}

The following script prints a dvi file on \verb+lwd+ (our double-sided
laser printer):

\begin{flushleft}
\label{dviprint_double_sided_start}
\noindent {\small {\sl $<<$dviprint\_double\_sided (1/1)$>>$} := }\\*
\tt
\noindent\makebox[0pt][r]{\makebox[4.5em][l]{601:}}%
\verb&#! /bin/sh&\\
\noindent\makebox[0pt][r]{\makebox[4.5em][l]{602:}}%
\verb&dvips -o tmp123.ps $1&\\
\noindent\makebox[0pt][r]{\makebox[4.5em][l]{603:}}%
\verb&lpr -P lwd tmp123.ps&\\
\noindent\makebox[0pt][r]{\makebox[4.5em][l]{604:}}%
\verb&rm tmp123.ps&\\
\end{flushleft}

\section*{Not done}

Line numbers from the original should be included in the generated
files. This is however not trivial since it can only be done between
function declarations, by adding \verb+-file(...)+ annotations to the
Erlang code, so the compiler gets the line numbers correct.

An emacs-mode should be written for editing Literate Erlang programs. As
a fix, the leading comment lines of EWEB~1.1 files allow you to begin a
file with something like:
\begin{verbatim}
 #    -*-latex-*-
 eweb1.1
 ...
\end{verbatim}
which would make Emacs go into \LaTeX{} mode.

Setting the ``executable'' flag on files should be done in an
OS-independent way, not by Unix \verb+chmod+.

\begin{thebibliography}{99}

    \bibitem{knuth}  Donald  E. Knuth and Silvio   Levy. {\sl The CWEB
system of Structured Documentation}. {\bf publisher??}

    \bibitem{noweb} Norman  Ramsey.  {\sl Literate-Programming   Tools
Need  Not Be   Complex}.  Research   report  CS-TR-351-91.   Princeton
University, 1992.

\end{thebibliography}

\end{document}
